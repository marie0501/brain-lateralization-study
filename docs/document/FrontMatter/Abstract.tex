\begin{resumen}
	El estudio analiza cómo los hemisferios cerebrales se especializan en procesar frecuencias espaciales de est\'imulos visuales, mediante modelos de campos receptivos poblacionales y modelos lineales mixtos. Se enfoca en el período preferido de los vértices corticales en diferentes áreas visuales, explorando su relación con la excentricidad visual y su influencia en la lateralización hemisférica. Los resultados revelan que el hemisferio derecho tiende a procesar frecuencias más bajas que el hemisferio izquierdo en áreas visuales superiores, aunque esta tendencia no se observa en áreas visuales primarias.  Este trabajo favorece la comprensión de cómo se procesa y representa la información visual, contribuyendo significativamente al conocimiento sobre la especialización funcional de los hemisferios en distintos niveles de procesamiento visual.
\end{resumen}

\begin{abstract}
	The study examines how the cerebral hemispheres specialize in processing spatial frequencies of visual stimuli using models of population receptive fields and mixed linear models. It focuses on the preferred period of cortical vertices in different visual areas, exploring its relationship with visual eccentricity and its influence on hemispheric lateralization. The results reveal that the right hemisphere tends to process lower frequencies than the left hemisphere in higher visual areas, although this trend is not observed in primary visual areas. This work enhances our understanding of how visual information is processed and represented, making a significant contribution to our knowledge of the functional specialization of the hemispheres at different levels of visual processing.
\end{abstract}