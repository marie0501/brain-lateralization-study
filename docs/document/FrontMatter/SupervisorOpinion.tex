\begin{opinion}
    La división de funciones entre los dos hemisferios cerebrales ha sido uno de los temas más intensamente estudiados en las neurociencias desde hace dos siglos, en cuanto a los procesos visuales. Esto ha generado un volumen valioso de datos psicológicos y clínicos. Con el desarrollo de los métodos modernos de neuroimágenes, se ha obtenido aún más información. Sin embargo, estos estudios no han generado modelos cuantitativos que puedan explicar cómo se diferencian los dos hemisferios. Hay más datos que teoría sólida. El lenguaje natural para formular modelos apropiados parece estar en el terreno de redes neurales convolucionales profundas, las cuales deben su reciente éxito práctico en diversos campos tecnológicos a su imitación al sistema visual humano. El desarrollo de la modelación de los campos receptivos poblacionales a partir de datos de resonancia magnética funcional abre el camino a modelos de redes neurales convolucionales m\'as realistas.  La tesis de Mari\'e del Valle Reyes sería un primer paso en examinar con campos receptivos poblacionales la lateralización hemisférica de los procesos visuales.
    
    Mari\'e del Valle Reyes es una alumna muy destacada, trabajando con entusiasmo de forma incansable. La estudiante ha demostrado una integración valiosa de habilidades en matemáticas aplicadas a la neurociencia en su tesis de diploma. Ha combinado una capacidad de programación en varios lenguajes (como Python, R, y Matlab) con el aprendizaje rápido de modelos estadísticos avanzados, y el aprendizaje de temas de neurociencias. Su capacidad para traducir conceptos matemáticos complejos en herramientas aplicables a la comprensión de procesos neurales muestra un enfoque interdisciplinario y original. Se han obtenido resultados importantes ya casi maduros para publicar en poco tiempo gracias a su tenacidad y entrega al trabajo. Su investigación es una contribución prometedora en la convergencia de las matemáticas y las neurociencias. Ha sido un gusto trabajar con ella.
\end{opinion}