\chapter{Discusi\'on}\label{chapter:discussion}

En este estudio, se confirma que el tamaño de los pRF y el período preferido crecen con la excentricidad en el campo visual. Es interesante destacar que la pendiente de este crecimiento es más pronunciada en regiones superiores de la vía visual. El hallazgo novedoso de esta tesis es que la preferencia de per\'iodo es distinta en los dos hemisferios cerebrales. En particular, se observó que en cuatro áreas visuales específicas (hV4, VO1, TO2, V3b), el período preferido de los vértices corticales era mayor en el hemisferio derecho en comparación con el izquierdo. Esto indica que las frecuencias espaciales preferidas por el hemisferio izquierdo son m\'as altas que las preferidas por el hemiferio derecho. Este efecto, no fue evidente en áreas visuales primarias (V1-V3) ni en algunas áreas de orden superior (por ejemplo, LO2).

Este trabajo es pionero al comparar la diferencia de selectividad para frecuencias espaciales entre los hemisferios cerebrales, en áreas visuales superiores. Es de notar que los pocos trabajos que examinan la diferencia de selectividad para frecuencias espaciales entre los dos hemisferios, solo lo hacen en áreas tempranas (V1-V3). Por ejemplo, en [\cite{broderick_mapping_2022}] no se analiz\'o la lateralizaci\'on hemisf\'erica en V1 y en [\cite{aghajari_population_2020}], se vi\'o la diferencia en varios cuadrantes del campo visual, pero  no se tuvieron en cuenta an\'alisis en \'areas visuales superiores. En este \'ultimo estudio, no se encontraron diferencias, lo cual concuerda con nuestros resultados, que indican que la especialización hemisf\'erica debe surgir después de las etapas más tempranas de la vía visual. 

Por tanto, nuestro trabajo ofrece la primera propuesta de un modelo explicativo para la especialización hemisférica en la percepción global/local. Específicamente nuestro estudio sugiere que la segunda etapa del modelo del Doble Filtraje por Frecuencias se debe a diferencias en la selectividad a frecuencias espaciales de los hemisferios cerebrales, debido a su organización en campos receptivos con propiedades distintas.  Falta explicar cómo se logra la primera etapa de este modelo que postula un ajuste del rango de frecuencias de trabajo según el tipo de estímulo. Se sugiere la necesidad de explorar cómo la preferencia por frecuencias espaciales puede variar al cambiar la atención del sujeto y cómo estas propiedades pueden adaptarse a diferentes estímulos.

Estas ideas se pueden explorar con redes neuronales convolucionales profundas. En trabajos realizados en CNEURO, fueron entrenadas redes para reconocer letras [Garea, comunicación personal]. Estas redes se construyeron con múltiples canales que diferían en el tamaño y la selectividad en frecuencias de filtros de Gabor, como primer paso del procesamiento de las imágenes con letras. Los filtros de Gabor son el mejor modelo para describir el campo receptivo de células visuales o poblaciones neuronales [\cite{kriegeskorte_understanding_2011}]. Cuando se entrenan estas redes para discriminar im\'agenes que tienen niveles globales y locales, eliminar los filtros de baja frecuencia impide la percepción de letras globales, y eliminar los filtros de alta frecuencia impide la percepción de letras locales. Sería interesante construir redes incorporando los parámetros fisiológicos medidos en este estudio y hacer una simulación más realista.
