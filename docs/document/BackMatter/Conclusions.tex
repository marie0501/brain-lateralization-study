\begin{conclusions}	
	
		
	En este trabajo, se alcanzaron objetivos fundamentales que profundizan nuestra comprensión de la especialización hemisférica del cerebro en el procesamiento de frecuencias espaciales en la percepción visual.
	 \begin{itemize}
	 	\item[1.] Se aplicó el modelo desarrollado por [\cite{broderick_mapping_2022}], que estima el período preferido (el inverso de la frecuencia espacial preferida) de los vértices corticales en las áreas visuales: V1, V2, V3, hV4, VO1, VO2, LO1, LO2, TO1, TO2, V3b y V3a. Esta metodología proporcionó una medida comparativa para analizar la lateralización hemisférica en la percepción de frecuencias espaciales de estímulos visuales.
	 	\item  [2.] Se utiliz\'o un modelo lineal mixto para interpretar los resultados del período preferido estimado, considerando su relación con el hemisferio cerebral y la excentricidad visual, as\'i como las diferencias individuales entre sujetos y los estímulos visuales presentados en el experimento. Los resultados de este modelo sirvieron para evaluar la hipótesis de nuestro estudio.
	 	
	 	\item [3.] 	A través del análisis de los resultados obtenidos con el modelo lineal mixto, se observ\'o que el período preferido de los vóxeles varía entre los hemisferios cerebrales en diversas áreas visuales. Esto respalda la hipótesis de una lateralización hemisférica. Además, se not\'o que, en las áreas visuales donde se evidencia esta diferencia (ej. hV4), el período preferido es más grande en el hemisferio derecho, indicando una inclinación de este hemisferio hacia frecuencias espaciales bajas. Estos hallazgos están en consonancia con datos neurofisiológicos y neuropsicológicos previos [\cite{flevaris_spatial_2016}] sobre la lateralización hemisférica.
	 	
	 \end{itemize}

	Por tanto, se concluye que los m\'etodos utilizados y los resultados de este estudio constituyen aportes significativos al campo de las neurociencias, especialmente a la comprensión de la selectividad de los hemisferios cerebrales a diferentes frecuencias espaciales en la percepción visual. Los  hallazgos encontrados no solo profundizan nuestro conocimiento del procesamiento visual, sino que también destacan la complejidad y adaptabilidad del cerebro en la interpretación del entorno visual.
	
	 



	
         
         
\end{conclusions}
