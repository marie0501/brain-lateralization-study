\chapter*{Introducción}\label{chapter:introduction}
\addcontentsline{toc}{chapter}{Introducción}

El cerebro humano es un sistema complejo que controla y regula la mayoría de las funciones del cuerpo y de la mente. Este órgano desempeña un papel esencial en la percepción.  y el procesamiento de la información, siendo el encargado de recibir, interpretar y responder a los estímulos del entorno, lo cual permite a los individuos interactuar de manera efectiva con el mundo. La información visual, 

Hay evidencias solidas de que en la percepción visual, los dos hemisferios cerebrales se especializan en funciones diferentes, a pesar su colaboracion estrecha. Esto se ha evidenciado respecto a la organizacion jerarquica de los escenas complejas, en las cuales se pueden definir distintos niveles organizativos.  Por ejemplo un acara es el nivel global, y los rasgos como ojos y boca son el nivel local. O un arbol es el nivel global y las ramas son el nivel local.

 Se ha observado [\cite{flevaris_spatial_2016}] que el hemisferio derecho muestra una especialzaion hacia lo aspectos globales  y el hemisferio izquierdo muestra una especialzacion para los aspectos locales. No obstante, cualquier teoría que se derive de estas consideraciones debe tener en cuenta que distintos aspectos de un estímulo visual pueden ser percibidos como globales en un contexto y como locales en otro. Por ejemplo, la percepción de un árbol puede considerarse global en comparación con una sola hoja, pero local en relación con un bosque.

Una hipótesis dominante sobre cómo se diferencian los niveles globales y locales de las imagenes en el  cerebro es en términos de las frecuencias espaciales de la mismas. En el contexto visual, las frecuencias espaciales se refieren al número de ciclos de un estímulo visual por unidad de ángulo visual, y es una propiedad importante de los estímulos visuales a la que las neuronas son sensibles. Estudios realizados [\cite{flevaris_spatial_2016}] muestran que las frecuencias espaciales más bajas se asocian comúnmente con aspectos globales en el hemisferio derecho, mientras que las SF más altas se asocian con aspectos locales en el hemisferio izquierdo. 



%La teoría DFF (Double Filtering by Frequency)\todo{citar articulo} de especialización hemisférica representa un mecanismo flexible que permite adaptarse a la tarea visual específica y al contexto en cuestión. Esta, postula que el procesamiento de frecuencias espaciales (SF) sigue dos etapas distintas. En la primera etapa, la atención elige un rango de SF de los espectros entrantes que sea más apropiado para la tarea visual actual. El rango de SF seleccionado se env\'ia de manera uniforme a ambos hemisferios del cerebro. En la segunda etapa, se manifiestan las diferencias entre los hemisferios. El hemisferio derecho funciona como un filtro de paso relativamente bajo, dando énfasis a las frecuencias más bajas dentro del rango inicialmente seleccionado, mientras que el hemisferio izquierdo opera como un filtro de paso relativamente alto, resaltando la información de las frecuencias más altas. De esta manera, una porci\'on particular del espectro SF puede ser preferida por el hemisferio derecho en un caso y por el hemisferio izquierdo en otro caso, seg\'un las caracter\'isticas del est\'imulo y la tarea en cuesti\'on.


La medición de los campos receptivos de la población (pRF) [\cite{dumoulin_population_2008}] mediante resonancia magnética funcional (fMRI) constituye una posible herramienta para evaluar y medir la lateralización hemisférica. Los pRFs constituyen modelos cuantitativos que predicen la actividad neuronal colectiva en un vóxel de fMRI en función de la selectividad de la respuesta a la posición del estímulo en el espacio visual. Estos modelos suelen estimar la posición y el tamaño de la sección del campo visual que afecta a un vóxel específico. La lateralización hemisférica puede estar vinculada a las propiedades de los pRfs.

Por otra parte, se ha establecido que las neuronas en la corteza visual de los primates (específicamente enla corteza visual primaria) muestran una sintonización con la frecuencia espacial, la cual depende de su ubicación en el campo visual. Existen modelos que determinan la frecuencia espacial preferida de los v\'ertices corticales o conjuntos de v\'ertices corticales de diferentes áreas visuales ([\cite{aghajari_population_2020}], [\cite{broderick_mapping_2022}]). Dada la sensibilidad variable a diversas frecuencias espaciales de estímulos visuales en los hemisferios cerebrales, una estrategia para cuantificar la lateralidad hemisférica podría ser la comparación de la frecuencia espacial preferida entre v\'ertices corticales homólogos en ambos hemisferios cerebrales.

El propósito del presente trabajo consiste en emplear modelos de pRF y modelos encargados de estimar la frecuencia preferida de los v\'ertices corticales, con el objetivo de evaluar la lateralización hemisférica de sus propiedades en el cerebro humano. La hipótesis que guía nuestra investigación sostiene que se postula que las frecuencias preferidas de los v\'ertices corticales en el hemisferio derecho deberían ser menores que las de los v\'ertices corticales homólogos en el hemisferio izquierdo. Se estipula, además, que esta diferencia debería ser más pronunciada en áreas superiores de las rutas  visuales.

% Cabe destacar que la elección de esta línea de investigación se fundamenta en la necesidad de profundizar en la comprensión de la organización hemisférica en el procesamiento visual, brindando así aportes significativos al conocimiento en el ámbito neurocientífico.

En el marco de la presente investigación, se utilizará un enfoque basado en modelos estadísticos avanzados, específicamente un modelo lineal mixto. Este modelo permitirá abordar la complejidad inherente a la variabilidad interindividual, proporcionando así una evaluación robusta de las diferencias observadas.


El objetivo general de este estudio es analizar la lateralización hemisférica en el procesamiento visual del cerebro humano, centrándose específicamente en la comparación de las frecuencias espaciales preferidas de v\'ertices corticales hom\'ologos en el hemisferio derecho e izquierdo, utilizando datos de fMRI.

Para lograr el objetivo general del presente trabajo se
trazan los siguientes objetivos específicos:

\begin{itemize}
	\item Estudiar el estado del arte sobre los temas abarcados en el estudio.
	
	\item  Aplicar modelos de pRF para estimar los campos receptivos de la población en ambos hemisferios cerebrales.
	
	\item Implementar modelos estadísticos para evaluar las diferencias en las frecuencias preferidas de los v\'ertices corticales entre hemisferios.	
	
	\item Analizar la lateralización hemisférica en diferentes áreas visuales y niveles de procesamiento visual.
	
	\item Interpretar y discutir los hallazgos, para proporcionar un enfoque m\'as detallado sobre la organización hemisférica en el procesamiento visual.
	
		
\end{itemize}

En lo siguiente, esta tesis se divide en cuatro capítulos. El primero, Marco Teórico, establece el contexto conceptual al explorar teorías clave sobre la lateralización hemisférica, pRFs y mapas retinot\'opicos. En el segundo cap\'itulo, Materiales y Métodos se detallan las técnicas y procedimientos empleados, facilitando la replicación del estudio. Resultados, es el tercer cap\'itulo, donde se presentan y analizan los hallazgos, utilizando gráficos para visualizar la lateralización hemisférica y las frecuencias espaciales preferidas. Finalmente, en el capítulo cuatro, Discusión, se interpreta críticamente los resultados, se exploran sus implicaciones y limitaciones, y se proponen posibles direcciones para futuras investigaciones.


 
	




