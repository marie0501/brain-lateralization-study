\chapter*{Introducción}\label{chapter:introduction}
\addcontentsline{toc}{chapter}{Introducción}

	El cerebro humano se concibe como un sistema complejo que controla y regula la mayoría de las funciones del cuerpo y de la mente. Este \'organo desempeña un papel esencial en la percepción y el procesamiento de la información, debido a que es el encargado de recibir, interpretar y responder a los estímulos del entorno, lo cual permite a los individuos interactuar de manera efectiva con su mundo. La percepción implica la interpretación y organización de los estímulos sensoriales para formar una representación consciente de la realidad. El cerebro procesa la información visual, auditiva, táctil y otras modalidades sensoriales, integrándola para construir una experiencia coherente y significativa del entorno circundante.
	
	Los dos hemisferios cerebrales desempeñan papeles notablemente diferentes en la percepción visual a pesar de su estrecha interacción, evidenciándose una lateralización hemisférica. Esta asimetría funcional se ha documentado exhaustivamente para procesar aspectos globales y locales de estímulos visuales \cite{flevaris_spatial_2016}, donde el hemisferio derecho muestra un sesgo global y el hemisferio izquierdo muestra una preferencia local. Estos sesgos para los niveles local/global se han establecido a través de métodos psicofísicos \cite{brederoo_hemispheric_2017, brederoo_reproducibility_2019} y electrofisiológicos \cite{flevaris_attending_2014, iglesias-fuster_asynchronous_2015, jiang_neural_2005}.
	
	El sesgo global/local de los hemisferios derecho/izquierdo podría explicarse en términos de las frecuencias espaciales (SF) asociadas con diferentes niveles de un estímulo visual. Las frecuencias espaciales describen cómo varía la información visual en términos de patrones de luz y sombra en una escena. Estos patrones pueden involucrar detalles finos o cambios suaves en la luminosidad. En el contexto visual, las SF más bajas se asocian comúnmente con aspectos globales, mientras que las SF más altas se asocian con aspectos locales \cite{flevaris_spatial_2016}. En apoyo a esta idea, varios estudios han revelado un sesgo hacia SF más bajas en el hemisferio derecho y SF más altas en el hemisferio izquierdo \cite{flevaris_spatial_2016}. Esto implica que el hemisferio derecho puede ser más eficiente en procesar información visual relacionada con la configuración global de un estímulo, mientras que el hemisferio izquierdo podría destacarse en detalles locales más finos. Además, se han encontrado vínculos entre la selección atencional global/local y las frecuencias espaciales de los est\'imulos \cite{flevaris_spatial_2016}. Sin embargo, cualquier teoría basada en estas consideraciones debe tener en cuenta el hecho de que los mismos aspectos de un estímulo visual pueden ser globales en un contexto y locales en otro (por ejemplo, un árbol es global en relación con una hoja pero local en relación con un bosque). El papel de cualquier SF no es fijo sino que depende del contexto.	
	
	Una posible forma de evaluar y medir la lateralización hemisférica se presenta a través de la medición de los campos receptivos de la población (pRF) mediante resonancia magnética funcional (fMRI) \cite{dumoulin_population_2008, kay_principles_2018} . Los pRF constituyen modelos cuantitativos que describen la respuesta combinada de las neuronas dentro de un vóxel de fMRI (vértice cortical). Estos modelos suelen estimar la posición y el tamaño de la sección del campo visual que afecta a un vóxel específico \cite{wandell_computational_2015}. La medición de respuestas de frecuencia espacial (SF) en los pRF también se ha convertido en un enfoque relevante, especialmente en áreas visuales tempranas (área visual primaria, V2, V3), ofreciendo perspectivas adicionales sobre la sintonización de SF en relación con el tamaño del pRF. Investigaciones recientes \cite{aghajari_population_2020, broderick_mapping_2022} han demostrado que la sintonización de la frecuencia espacial de un pRF tiende a disminuir a medida que aumenta su tamaño, lo que sugiere una posible relación entre la lateralización hemisférica y las características de procesamiento de la información visual en el cerebro. Estas observaciones brindan una valiosa perspectiva para comprender cómo la lateralización hemisférica puede estar vinculada a las propiedades de los campos receptivos de la población.
	
	La investigación sobre las propiedades de los pRF en funci\'on del campo visual, se ha focalizado principalmente en las diferencias entre los cuadrantes superior e inferior en la corteza visual primaria (V1), donde no se han observado diferencias significativas entre los cuadrantes derecho e izquierdo. Un estudio encontró tamaños de pRF ligeramente más pequeños en el cuadrante izquierdo en comparación con los cuadrantes horizontales derechos de las \'areas visuales V2 y V3 y, nuevamente, no hubo diferencias en V1 \cite{silva_radial_2018}. Esta observación sugiere que la lateralidad de las propiedades de los pRF en áreas visuales intermedias y superiores no se ha estudiado exhaustivamente. La dificultad para identificar sitios homólogos entre hemisferios en estas regiones, donde los mapas retinotópicos son menos definidos, puede contribuir a la falta de investigación detallada en estas áreas.
	
	Existen varias bases de datos p\'ublicas de pRF \cite{benson_bayesian_2018,himmelberg_cross-dataset_2021}, que cubren amplias extensiones de la corteza cerebral, lo que hace posible la prueba de la lateralidad de las propiedades de pRF en \'areas visuales de orden intermedio o superior. Para examinar las diferencias derecha/izquierda en el tamaño del pRF, se pueden utilizar dos estrategias denominadas aquí \textbf{anatómicas} y \textbf{homotópicas}. El objetivo es comparar sitios corticales homólogos, pero como se mencionó anteriormente, definir "homólogo" presenta dificultades, especialmente para áreas de orden superior con respuesta visual.
	
	La estrategia anatómica mide las diferencias en las propiedades de pRF en un espacio donde las superficies corticales son aproximadamente simétricas en los hemisferios izquierdo y derecho. Esta simetría significa que cuando se considera el orden de los vértices, aquellos con el mismo rango en ambos hemisferios son aproximadamente homólogos.
	
	El enfoque homotópico compara los regiones corticales con la conectividad anatómica o funcional más fuerte entre los dos hemisferios, una conexión que indica que probablemente trabajen juntos. Por lo tanto, la lateralización de las propiedades de pRF se puede examinar con una parcelación basada en la conectividad de resonancia magnética funcional en estado de reposo o relacionada con la tarea que identifica pares de áreas corticales homotópicas.  
	
	El objetivo general de este estudio es determinar de manera sistemática si existen diferencias en las propiedades de los campos receptivos entre los hemisferios izquierdo y derecho en áreas visuales de orden intermedio y superior. Para alcanzar este propósito, se emplearán algoritmos computacionales especializados y pruebas estadísticas rigurosas. El análisis se llevará a cabo mediante la aplicación de estrategias previamente descritas a tres bases de datos pRF. La utilización de algoritmos computacionales permitirá la extracción y comparación de las propiedades de los campos receptivos en ambos hemisferios, mientras que la aplicación de pruebas estadísticas proporcionará una evaluación cuantitativa de la significancia de las posibles diferencias identificadas.
	
	Para lograr el objetivo general del presente trabajo se
	trazan los siguientes objetivos específicos:
	
	\begin{itemize}
		\item Estudio del estado del arte sobre el preprocesamiento de las resonancias magn\'eticas funcionales.
		\item Estudiar el estado del arte sobre los modelos de campos receptivos poblacionales.
		\item Estudiar los elementos te\'oricos de la lateralidad hemisf\'erica cerebral.
		\item Implementar y evaluar las estrategias concebidas para la examinaci\'on de las diferencias en ambos hemisferios cerebrales en el tama\~no de los pRF.
		
	\end{itemize}
	
	En lo siguiente, esta tesis se divide en cuatro capítulos.  En el Capítulo 2, titulado "Marco Teórico-Conceptual", se realiza un análisis detallado del estado actual de la ciencia y tecnología en las áreas relevantes de la esta investigaci\'on. En el Capítulo 3, titulado ''Concepción y Diseño de las Estrategias", se describe en detalle la metodología para abordar la investigación sobre la lateralidad hemisférica, incluyendo aspectos clave del enfoque analítico. Detalles técnicos de la implementación del sistema se presentan en el Capítulo 4, titulado ''Implementación y Experimentación". En este capítulo, se explora cualitativamente la validez de la solución implementada, aprovechando las herramientas disponibles. Se describen los métodos y técnicas utilizadas para evaluar la lateralidad hemisférica en áreas visuales de orden intermedio y superior, destacando los resultados y observaciones obtenidas durante la experimentación. En la parte del desenlace, se presenta el Capítulo 5, donde se exponen las conclusiones de la investigación. Se destacan los logros clave en relación con los objetivos planteados, proporcionando un resumen de los hallazgos más significativos. Además, se presentan recomendaciones que señalan futuras direcciones de investigación, brindando perspectivas para la continuidad del estudio sobre la lateralidad hemisférica y el procesamiento visual. Finalmente, la bibliografía utilizada para respaldar la base científica de la solución propuesta y los anexos complementarios se incluyen en secciones respectivas, facilitando la exploración de temas relacionados y proporcionando una base sólida para la validación y respaldo de la investigación realizada.
	


