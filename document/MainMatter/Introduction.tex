\chapter*{Introducción}\label{chapter:introduction}
\addcontentsline{toc}{chapter}{Introducción}

El cerebro humano se concibe como un sistema complejo que controla y regula la mayoría de las funciones del cuerpo y de la mente. Este \'organo desempeña un papel esencial en la percepción y el procesamiento de la información, debido a que es el encargado de recibir, interpretar y responder a los estímulos del entorno, lo cual permite a los individuos interactuar de manera efectiva con su mundo. La percepción implica, la interpretación y organización de los estímulos sensoriales para formar una representación consciente de la realidad. El cerebro procesa la información visual, auditiva, táctil y otras modalidades sensoriales, integrándola para construir una experiencia coherente y significativa del entorno circundante.

En la percepci\'on visual, los hemisferios cerebrales se especializan en funciones diferentes, a pesar de colaborar estrechamente en el procesamiento de la informaci\'on. Se ha observado [\cite{flevaris_spatial_2016}] que para procesar aspectos globales y locales de estímulos visuales, el hemisferio derecho muestra una inclinaci\'on hacia lo global y el hemisferio izquierdo muestra una preferencia local. No obstante, cualquier teoría que se derive de estas consideraciones debe tener en cuenta que distintos aspectos de un estímulo visual pueden ser percibidos como globales en un contexto y como locales en otro. Por ejemplo, la percepción de un árbol puede considerarse global en comparación con una sola hoja, pero local en relación con un bosque.

Una hipótesis dominante sobre cómo se diferencian los niveles globales y locales en el cerebro es en términos de sus frecuencias espaciales (SF)\todo{poner en ingl\'es}. En el contexto visual, las SF se refieren al número de ciclos de un estímulo visual por unidad de ángulo visual, y es una propiedad importante de los estímulos visuales a la que las neuronas son sensibles. Estudios realizados [\cite{flevaris_spatial_2016}] muestran que las SF más bajas se asocian comúnmente con aspectos globales en el hemisferio derecho, mientras que las SF más altas se asocian con aspectos locales en el hemisferio izquierdo. 



%La teoría DFF (Double Filtering by Frequency)\todo{citar articulo} de especialización hemisférica representa un mecanismo flexible que permite adaptarse a la tarea visual específica y al contexto en cuestión. Esta, postula que el procesamiento de frecuencias espaciales (SF) sigue dos etapas distintas. En la primera etapa, la atención elige un rango de SF de los espectros entrantes que sea más apropiado para la tarea visual actual. El rango de SF seleccionado se env\'ia de manera uniforme a ambos hemisferios del cerebro. En la segunda etapa, se manifiestan las diferencias entre los hemisferios. El hemisferio derecho funciona como un filtro de paso relativamente bajo, dando énfasis a las frecuencias más bajas dentro del rango inicialmente seleccionado, mientras que el hemisferio izquierdo opera como un filtro de paso relativamente alto, resaltando la información de las frecuencias más altas. De esta manera, una porci\'on particular del espectro SF puede ser preferida por el hemisferio derecho en un caso y por el hemisferio izquierdo en otro caso, seg\'un las caracter\'isticas del est\'imulo y la tarea en cuesti\'on.


La medición de los campos receptivos de la población (pRF) \todo{referenciar?} mediante resonancia magnética funcional (fMRI) constituye una posible herramienta para evaluar y medir la lateralización hemisférica. Los pRFs constituyen modelos cuantitativos que predicen la actividad neuronal colectiva en un vóxel de fMRI en función de la selectividad de la respuesta a la posición del estímulo en el espacio visual. Estos modelos suelen estimar la posición y el tamaño de la sección del campo visual que afecta a un vóxel específico. La lateralización hemisférica puede estar vinculada a las propiedades de los pRfs.

Por otra parte, se ha establecido que las neuronas en la corteza visual de los primates (específicamente en el área V1) muestran una sintonización con la frecuencia espacial, la cual depende de su ubicación en el campo visual. Existen modelos que determinan la frecuencia espacial preferida de los vóxeles o conjuntos de v\'oxeles de diferentes áreas visuales ([\cite{aghajari_population_2020}], [\cite{broderick_mapping_2022}]). Considerando la sensibilidad de diferentes frecuencias espaciales de est\'imulos visuales en los hemisferios cerebrales, una metodología para medir la lateralidad hemisférica podría ser la comparación de la frecuencia espacial preferida de vóxeles homólogos en ambos hemisferios cerebrales. 

El propósito del presente trabajo consiste en emplear modelos de pRF y modelos encargados de estimar la frecuencia preferida de los v\'oxeles, con el fin de evaluar la lateralización hemisférica en el cerebro humano. La hipótesis que guía nuestra investigación sostiene que los tamaños de los pRFs en el hemisferio derecho, en comparación con los tamaños de los pRFs an\'alogos en el hemisferio izquierdo, deben ser mayores. Asimismo, se postula que las frecuencias preferidas de los v\'oxeles en el hemisferio derecho deberían ser menores que las de los v\'oxeles homólogos en el hemisferio izquierdo. Se estipula, además, que esta diferencia debería ser más pronunciada en áreas superiores del procesamiento visual.

% Cabe destacar que la elección de esta línea de investigación se fundamenta en la necesidad de profundizar en la comprensión de la organización hemisférica en el procesamiento visual, brindando así aportes significativos al conocimiento en el ámbito neurocientífico.

En el marco de la presente investigación, se utilizará un enfoque basado en modelos estadísticos avanzados, específicamente un modelo lineal generalizado mixto. Este modelo permitirá abordar la complejidad inherente a la variabilidad interindividual, proporcionando así una evaluación robusta de las diferencias observadas.

Adicionalmente, se llevará a cabo la simulación de los resultados empleando un modelo de red neuronal artificial. Este enfoque se ha diseñado como una herramienta complementaria para la representación en el sistema visual de los resultados obtenidos. La utilización de una red neuronal artificial permitirá no solo corroborar los hallazgos observados con el modelo lineal generalizado mixto, sino también proporcionar una perspectiva más detallada de las complejas interrelaciones subyacentes en la organización hemisférica del procesamiento visual.

El objetivo general de este estudio es analizar la lateralización hemisférica en el procesamiento visual del cerebro humano, centrándose específicamente en la comparación de los tamaños de pRF y las SP preferidas de v\'oxeles hom\'ologos en el hemisferio derecho e izquierdo, utilizando datos de fMRI.

Para lograr el objetivo general del presente trabajo se
trazan los siguientes objetivos específicos:

\begin{itemize}
	\item Aprender los elementos te\'oricos de la lateralidad hemisf\'erica cerebral.
	
	\item Estudiar el estado del arte sobre los modelos de pRF y modelos de estimaci\'on de SF preferida de los v\'oxeles en \'areas visuales.
	
	\item Implementar y evaluar las estrategias concebidas para el an\'alisis de las diferencias en ambos hemisferios cerebrales, en el tama\~no de los pRF y la preferencia de frecuencia espacial de v\'oxeles hom\'ologos.
	
\end{itemize}
\todo{Esto no es seguro}
En lo siguiente, esta tesis se divide en cuatro capítulos. El primero refiere un estudio sobre trabajos relacionados al tema presentado. En el segundo, se ofrecerá una visión general de los conceptos teóricos asociados a mapas retinot\'opicos, modelos de pRF y modelos de estimaci\'on de frecuencia espacial preferida. En tercer cap\'itulo, se describe en detalle la metodología para abordar la investigación sobre la lateralidad hemisférica, incluyendo aspectos clave del enfoque analítico. Detalles técnicos de la implementación del sistema se presentan en el cuarto cap\'itulo, donde se explora cualitativamente la validez de la solución implementada, aprovechando las herramientas disponibles. Se describen los métodos y técnicas utilizadas para evaluar la lateralidad hemisférica en áreas visuales, destacando los resultados y observaciones obtenidas durante la experimentación.
 %En la parte del desenlace, se presenta el Capítulo 5, donde se exponen las conclusiones de la investigación. Se destacan los logros clave en relación con los objetivos planteados, proporcionando un resumen de los hallazgos más significativos. Además, se presentan recomendaciones que señalan futuras direcciones de investigación, brindando perspectivas para la continuidad del estudio sobre la lateralidad hemisférica y el procesamiento visual. Finalmente, la bibliografía utilizada para respaldar la base científica de la solución propuesta y los anexos complementarios se incluyen en secciones respectivas, facilitando la exploración de temas relacionados y proporcionando una base sólida para la validación y respaldo de la investigación realizada.
	



%Amplia evidencia ha demostrado que los hemisferios son asimétricos en el procesamiento de información más global y más local en un estímulo visual: estudios cognitivos (Hillger y Koenig, 1991; Martin, 1979; Sergent, 1982), computacionales (Ivry y Robertson, 1998), neuropsicológico (Delis et al., 1986; Lamb et al., 1990; Robertson y Delis, 1986; Robertson y Lamb, 1991; Robertson et al., 1988) y neurofisiológico (Fink et al., 1997; Iglesias-Fuster et al., 2015; Han y Chen, 1996; Han et al., 2002, 2000; Heinze et al., 1998; Martinez et al., 1997; Weissman y Woldorff, 2005) han encontrado diferencias hemisféricas funcionales en global versus procesamiento local, con el hemisferio derecho del cerebro demostrando un sesgo global y el hemisferio izquierdo demostrando un sesgo local (Ivry y Robertson, 1998; Robertson e Ivry, 2000). La interpretación de estos hallazgos es más complicada si consideramos el hecho de que el nivel en el que se percibe un objeto en particular puede variar. Por ejemplo, cuando vemos un bosque, un árbol en el bosque es un elemento local en la escena global, pero si centramos la atención en el árbol, se convierte en el nivel global, mientras que las hojas, las ramas y el tronco son elementos locales. Por lo tanto, cualquier teoría del procesamiento global y local debe proporcionar una explicación de cómo el sistema perceptivo define los niveles global y local para dar lugar a esta asimetría hemisférica funcional. Una hipótesis dominante sobre cómo se diferencian los niveles globales y locales en el cerebro es en términos de sus frecuencias espaciales (FS).
