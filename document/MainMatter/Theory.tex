\chapter{Marco Te\'orico}\label{chapter:theory}

\subsection*{Mapa Retinot\'opico Bayesiano}
En el contexto del estudio de Benson et al. (2012), el "ángulo polar" y la "excentricidad" se refieren a medidas utilizadas para describir la localización de las respuestas neuronales en el córtex visual. El ángulo polar indica la posición angular de un estímulo visual en el campo visual, como su posición relativa de arriba a abajo o de izquierda a derecha. La excentricidad, por otro lado, se refiere a la distancia de un estímulo visual desde el punto de fijación central en el campo visual, es decir, qué tan lejos está del centro de la visión. Estas medidas son clave para entender cómo el córtex visual mapea la información visual del mundo externo.
Explicar en que consiste
\subsection*{pRF}
Explicar la modelacion de los pRF
\subsection*{Periodo Preferido}
Ver modelo de periodo preferido de BRoderick