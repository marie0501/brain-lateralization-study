\chapter{Estado del Arte}\label{chapter:state-of-the-art}


\section*{Mapas Retinot\'opicos}

En \cite{benson_retinotopic_2012} se desarroll\'o de un algoritmo automático y determinista para ajustar un modelo retinotópico al área anatómicamente definida V1 en ausencia del usuario.

En \cite{benson_bayesian_2018} se desarrolla un nuevo enfoque de mapeo bayesiano que combina la observación, unas
mediciones retinotópicas del sujeto a partir de pequeñas cantidades de tiempo de fMRI, con un previo, un 
atlas retinotópico. Este proceso dibuja automáticamente límites de área, corrige discontinuidades en los
mapas medidos y predice datos de validación con mayor precisión que un atlas solo o conjuntos independiente
de datos.

se aborda la organización del córtex visual humano, que está compuesto por múltiples mapas retinotópicos. La caracterización precisa de la disposición de estos mapas en la superficie cortical es fundamental para numerosas investigaciones en neurociencia visual. En lugar de depender únicamente del análisis voxel a voxel de datos de fMRI, que mapea solo una porción del campo visual y está limitado por ruido de medición y evaluación subjetiva de límites, se presenta un novedoso enfoque de mapeo bayesiano. Este método combina observaciones, es decir, mediciones retinotópicas de un sujeto a partir de pequeñas cantidades de tiempo de fMRI, con un conocimiento previo, es decir, un atlas retinotópico aprendido. Este proceso automatiza la delimitación de áreas, corrige discontinuidades en los mapas medidos y predice datos de validación de manera más precisa que un atlas solo o conjuntos de datos independientes solos. Esta innovadora metodología no solo mejora la precisión del mapeo retinotópico, sino que también posibilita el análisis automático de grandes conjuntos de datos de fMRI y cuantifica las diferencias en las propiedades del mapa en función de la salud, el desarrollo y las variaciones naturales entre individuos.

\section*{pRF}
En \cite{carvalho_micro-probing_2020}, se aborda la caracterización de las propiedades del campo receptivo (RF), un aspecto fundamental para comprender las bases neurales del comportamiento sensorial y cognitivo. Se presenta una técnica innovadora denominada micro-probing (MP), diseñada para una caracterización detallada y en gran medida libre de suposiciones de múltiples campos receptivos (pRFs) dentro de un voxel. A diferencia de enfoques actuales que requieren numerosas suposiciones a priori y no pueden revelar propiedades inesperadas, como RFs fragmentados o subpoblaciones, MP supera estas limitaciones al permitir la detección de formas y propiedades inesperadas de RF, mejorando el detalle espacial de análisis de datos. MP utiliza modelos Gaussianos de tamaño fijo para muestrear eficientemente todo el espacio visual, creando mapas de sondas detallados, y posteriormente, se derivan pRFs a partir de estos mapas. La efectividad de MP se demuestra mediante simulaciones y al cartografiar campos visuales de participantes saludables y de un grupo de pacientes con RFs anómalos debido a un trastorno congénito de la vía visual. Sin requerir estímulos específicos ni modelos adaptados, MP mapeó los pRFs bilaterales característicos de observadores con albinismo. Además, en observadores sanos, MP reveló que los voxels pueden capturar la actividad de múltiples subpoblaciones de RFs que muestrean regiones distintas del campo visual. En resumen, MP proporciona un marco versátil para visualizar, analizar y modelar, sin restricciones, los diversos RFs de subpoblaciones corticales en la salud y la enfermedad.

En \cite{dumoulin_population_2008} se computa un modelo del pRF y estima el campo visual asi como otras propiedades como el tamanno y la lateralidad del campo receptivo. Los mapas obtenidos del campo visual con este metodos son mejores que los resusltados del metodo convencional. Se comparan los mapas obtenidos con el pRF con los mapas obtenidos con metodos convencionales utilizando un coeficiente de correlacion.

En \cite{zeidman_bayesian_2018}, se presenta un marco probabilístico (Bayesiano) y una caja de herramientas de software asociada para mapear campos receptivos poblacionales (pRFs) basados en datos de fMRI. Este enfoque genérico está diseñado para funcionar con estímulos de cualquier dimensión y se demuestra y valida en el contexto de mapeo retinotópico bidimensional (2D). El marco permite al experimentador especificar modelos generativos (de codificación) de series temporales de fMRI, en los cuales los estímulos experimentales ingresan a un modelo pRF de actividad neural, que a su vez impulsa un modelo no lineal de acoplamiento neurovascular y respuesta dependiente del nivel de oxígeno en sangre (BOLD). Los parámetros neuronales y hemodinámicos se estiman conjuntamente, voxel por voxel o en regiones de interés, utilizando un algoritmo de estimación bayesiana (Laplaciano variacional). Esto aporta varias contribuciones novedosas al modelado de campos receptivos. Se estiman la varianza/covarianza de los parámetros, lo que permite representar correctamente la incertidumbre sobre el tamaño y la ubicación del pRF. Se tiene en cuenta la variabilidad en la respuesta hemodinámica en todo el cerebro. Además, el marco introduce pruebas formales de hipótesis para el análisis de pRF, lo que permite evaluar modelos competitivos basados en su evidencia de modelo logarítmico (aproximado por la energía libre variacional), que representa el equilibrio óptimo entre precisión y complejidad. Mediante simulaciones y datos empíricos, se encontró que los parámetros comúnmente utilizados para representar el tamaño del pRF y la escala neuronal están fuertemente correlacionados, y este hecho se tiene en cuenta en los métodos Bayesianos descritos al realizar inferencias. Se utilizó el marco para comparar la evidencia de seis variantes de modelos pRF utilizando datos de fMRI funcional a 7 T, y se encontró que un modelo circular de Diferencia de Gaussianas (DoG) fue la mejor explicación para nuestros datos en general. Se espera que este marco resulte útil para mapear espacios de estímulos con cualquier número de dimensiones sobre la anatomía del cerebro.

La resonancia magnética funcional (fMRI) mide de manera no invasiva la actividad cerebral humana con una resolución de milímetros. Los científicos emplean diversas estrategias para aprovechar las extraordinarias oportunidades que ofrece la fMRI. En este contexto, se destaca el enfoque computacional de neuroimagen en la corteza visual humana, cuyo objetivo es construir modelos predictivos de las respuestas neuronales a partir de estímulos y tareas específicas. Se resalta la investigación activa en el uso de modelos de campo receptivo poblacional (pRF) para caracterizar las respuestas de la corteza visual humana a una variedad de estímulos, en distintas tareas y en diversas poblaciones de sujetos. Este enfoque ha mostrado avances significativos, permitiendo una comprensión más profunda de la organización funcional de la corteza visual mediante la predicción de respuestas neuronales a estímulos y tareas específicas.\cite{wandell_computational_2015}

\section*{Modelos de frecuencia espacial}

En \cite{broderick_mapping_2022} se realiza un estudio que buscó caracterizar la sintonización de las neuronas en la corteza visual de primates, específicamente en el área V1, en relación con la frecuencia espacial. La frecuencia espacial se refiere al número de ciclos de un estímulo visual por unidad de ángulo visual, y es una propiedad importante de los estímulos visuales a la que las neuronas en el sistema visual son sensibles.

El estudio utilizó resonancia magnética funcional (fMRI) para medir las respuestas de las neuronas en la corteza visual primaria de los seres humanos. Los investigadores emplearon un conjunto novedoso de estímulos que consistían en rejillas sinusoidales dispuestas en coordenadas log-polares, que incluían geometrías circulares, radiales y espirales. Este conjunto de estímulos les permitió investigar la sintonización de la frecuencia espacial en diferentes orientaciones y ubicaciones en el campo visual.

En el estudio \cite{aghajari_population_2020}, se investigó la selectividad de las neuronas en la corteza visual temprana para estadísticas básicas de imágenes, centrándose en la frecuencia espacial. Utilizando una aproximación novedosa de análisis de fMRI basada en modelos, se buscó una estimación eficiente de la sintonización de la frecuencia espacial de la población (pSFT) para voxels individuales. Durante la adquisición de respuestas BOLD, los sujetos observaron estímulos generados por el filtrado de ruido blanco con cambios periódicos en la frecuencia central. Los resultados revelaron que, conforme aumenta la excentricidad en cada área visual, la frecuencia espacial máxima de la pSFT disminuye. Además, se encontró que la amplitud de la pSFT está correlacionada con la excentricidad y la frecuencia espacial máxima, mostrando que poblaciones con frecuencias máximas más bajas tienen amplitudes más amplias en escala logarítmica, y esta relación se invierte en escala lineal. Estos hallazgos contribuyen a una comprensión más profunda de cómo las neuronas en la corteza visual temprana responden a diferentes frecuencias espaciales, destacando la eficacia del enfoque basado en modelos en el análisis preciso y rápido de datos fMRI en este contexto.