\chapter{Estado del Arte}\label{chapter:state-of-the-art}

\subsubsection{Asimetr\'ia hemisf\'erica en humanos en la percepci\'on visual y relaci\'on con frecuencia espacial de los est\'imulos visuales}

La asimetría hemisf\'erica indica que cada hemisferio del cerebro tiene especializaciones únicas en el procesamiento de la información visual, contribuyendo de manera distinta a la comprensión y percepción del mundo visual. A lo largo de las décadas, diversas investigaciones han revelado que mientras el hemisferio izquierdo es más eficaz en el procesamiento de detalles finos y de alta frecuencia, el derecho sobresale en la percepción de patrones globales y de baja frecuencia.

En [\cite{flevaris_spatial_2016}] se profundiza en la asimetría hemisférica del cerebro y la relación entre la percepción global/local y el procesamiento de frecuencias espaciales (SF) en estímulos visuales. Se revisan investigaciones que sugieren que las frecuencias bajas (LSF) se asocian con la percepción global, procesadas predominantemente por el hemisferio derecho, mientras que las altas (HSF) se vinculan con la percepción local, manejadas por el hemisferio izquierdo. 

El art\'iculo expone que existen diversos estudios cognitivos, neuropsicológicos y neurofisiológicos donde se han encontrado diferencias hemisféricas funcionales en el procesamiento global/local, con el hemisferio derecho del cerebro demostrando un sesgo global y el hemisferio izquierdo demostrando un sesgo local. Por ejemplo, al presentar visualizaciones de letras de Navon ([\cite{navon_forest_1977}]) en los campos visuales derecho e izquierdo los resultados revelan que los participantes identifican más rápido la letra global cuando se presenta en el campo visual izquierdo (proyectado al hemisferio derecho) y más rápido la letra local cuando se presenta en el campo visual derecho (proyectado al hemisferio izquierdo). Este sesgo también se ve respaldado por investigaciones con pacientes con lesiones cerebrales, donde se identificó que un grupo de pacientes con lesiones centradas en el hemisferio derecho experimentaban dificultades selectivas en la percepción de elementos globales en estímulos de letras de Navon, mientras que otro grupo de pacientes con lesiones centradas en el hemisferio izquierdo manifestaba selectivamente problemas para percibir elementos locales. Tambi\'en se expone que estudios de fMRI \todo{ver si escribi abreviatura antes} y de electroencefalograma (EEG) respaldaron esta asimetría.

Por otra parte, se explica que la relación entre las LSF y la percepción global, así como entre las HSF y la percepción local, se evidenció en experimentos donde los sujeros participaron en la detección de rejillas de SF en tareas global/local. Se observó que eran más eficientes en la detección de rejillas LSF después de prestar atención a una forma global, mientras que eran más hábiles en la detección de rejillas HSF después de focalizar su atención en las formas locales. Además, se han demostrado que la identificación de una rejilla sinusoidal con líneas ''más gruesas'' respalda un proceso del hemisferio derecho sesgado hacia los LSF, mientras que la identificación de líneas ''más delgadas'' respalda un proceso del hemisferio izquierdo sesgado hacia los HSF. 



\subsubsection*{Mapas Retinot\'opicos}
Un mapa retinotópico es la organización de la corteza visual en el cerebro que refleja la disposición espacial de la retina. Se cumple que puntos cercanos en la retina se corresponden con puntos cercanos en la corteza visual, lo cual implica que la disposición espacial de las imágenes en la retina se conserva en la representación cortical. Estos mapas son fundamentales para entender cómo el cerebro procesa la información visual y cómo se traducen las imágenes visuales en percepciones.\todo{estoy diciendo lo mismo dos veces}

En [\cite{wandell_imaging_2011}] se aborda el mapeo retinotópico en el cerebro humano utilizando fMRI. Se enfoca en los avances realizados en los últimos 25 años en la comprensión de los mapas de campos visuales en el cerebro humano, destacando el progreso significativo en las tecnologías de fMRI y en los métodos experimentales.

En [\cite{benson_retinotopic_2012}] se examina la organización retinotópica de V1 \todo{ver si lo puse en intro}, demostrando que la topología superficial del cerebro puede predecir con precisión la función retinotópica interna. Con estudios de fMRI, se hallaron en los participantes las medidas utilizadas para describir la localización de las respuestas neuronales en la corteza visual (ángulo polar y excentricidad), y se desarroll\'o un modelo algebraico para ajustar estos datos. 

En [\cite{benson_bayesian_2018}] se introduce un enfoque de análisis bayesiano para mapear mapas retinotópicos en el cerebro humano. Se presenta un modelo que combina datos retinotópicos (ángulo polar y excentricidad) y un atlas retinotópico\todo{definici\'on atlas?} preexistente a través de inferencia bayesiana, mejorando así la precisión y completitud de los mapas retinotópicos. 


\subsubsection*{Modelos de Campo Receptivo de Población} 

El campo receptivo de población (pRF) es un concepto en neurociencia que describe un modelo que representa cómo un grupo de neuronas en una región específica del cerebro responde colectivamente a un estímulo visual. Este modelo permite entender mejor la actividad y la organización de la corteza visual, mostrando cómo diferentes áreas procesan información visual en conjunto, en lugar de enfocarse en las respuestas de neuronas individuales. Este modelo se ha utilizado para mapear la organización cortical,  revelar los efectos de la atención en el procesamiento visual y  mostrar diferencias en pacientes y poblaciones especiales.

En [\cite{dumoulin_population_2008}] se formula un modelo matemático del pRF utilizando una función Gaussiana que describe cómo la respuesta neuronal varía con la distancia desde el centro del campo receptivo. Incluye parámetros como la posición del centro del pRF, su tamaño, y la magnitud de la respuesta. La función Gaussiana se ajusta a los datos de actividad neuronal, obtenidos a través fMRI, para estimar las características del pRF en una población neuronal específica.

En [\cite{zuiderbaan_modeling_2012}] se introduce un modelo de pRF basado en la función Diferencia de Gaussianas (DoG), mejorando la capacidad de representar respuestas inhibitorias y la supresión periférica en la corteza visual.

En [\cite{kay_compressive_2013}] se desarrolla un modelo de pRF con una no linealidad estática compresiva, es decir, reduce la amplitud de las respuestas a medida que aumenta la intensidad del estímulo. Este modelo explica que la respuesta total a estímulos múltiples es menor que la suma de respuestas individuales.

En [\cite{amano_visual_2009}] se analizan los mapas de campo visual y los tamaños de los pRF en el complejo MT+\todo{poner que es MT} humano, usando fMRI. Se hall\'o que el tamaño de los pRF aumenta progresivamente desde V1/2/3 hasta LO-1/2 y TO-1/2, siendo TO-2 el área con los pRF más grandes. También observaron que dentro de cada mapa, el tamaño de los pRF aumenta con la excentricidad. 

En [\cite{winawer_mapping_2010}] se examina la organización del mapa visual del \'area hV4 y la corteza occipital ventral a trav\'es de fMRI y técnicas de pRF. 

En [\cite{schwarzkopf_larger_2014}] se investigan los pRFs en personas con trastornos del espectro autista. 

En [\cite{wandell_computational_2015}] se destaca la utilizaci\'on de modelos pRFs para caracterizar las respuestas neurales a diferentes est\'imulos y tareas visuales y se enfoca en la aplicación de pRFs para entender la atención, la plasticidad y las diferencias en condiciones psiquiátricas y neurológicas.

En [\cite{kay_attention_2015}] se realiza un estudio que muestra que la atención aumenta la ganancia, el tamaño y la excentricidad de los pRF en áreas de alto nivel, pero no en áreas visuales tempranas.

En [\cite{welbourne_population_2018}] se analiza cómo la corteza visual humana procesa la información cromática a través de mediciones de campos receptivos de población (pRF) usando fMRI. 

En [\cite{benson_human_2018}] del Human Connectome Project, los pRF se utilizan para analizar la organización retinotópica de la corteza visual y subcortical en 181 adultos sanos. 

En de [\cite{himmelberg_cross-dataset_2021}] se comparan las propiedades de los pRFs de dos conjuntos de datos diferentes: uno de la Universidad de Nueva York (NYU) y otro del Human Connectome Project (HCP) [\cite{benson_human_2018}].  



\subsubsection*{An\'alisis de modelos de representaci\'on en el sistema visual humano}

Las representaciones en el sistema visual humano se refieren a cómo el cerebro interpreta y procesa la información visual que recibe desde la retina. Incluyen desde la percepción básica de formas, colores y movimientos, hasta la interpretación compleja de escenas, rostros y expresiones emocionales. Con el objetivo de entender mejor la visi\'on humana, se han desarrollado modelos que combinan las neurociencias, la psicología y la inteligencia artificial.

En [\cite{henriksson_spatial_2008}]  se midieron curvas de sintonización de frecuencia espacial en áreas como V1, V2, VP, V3, V4v, V3A, y V5+, utilizando fMRI. Se observ\'o que la frecuencia espacial óptima disminuye con el aumento de la excentricidad visual y varía según la ubicación retinotópica. Estos hallazgos apoyan la idea de que diferentes áreas visuales procesan la información visual en distintas escalas espaciales.

En [\cite{aghajari_population_2020}] se aplica un modelo log-Gaussiano para estimar la sintonización de frecuencia espacial a nivel de cada vóxel cerebral, revelando cómo varía la sensibilidad a la frecuencia espacial en la corteza visual inferior. Se miden respuestas a estímulos visuales de diferentes frecuencias espaciales, a trav\'es de fMRI. Los resultados indican que la frecuencia espacial preferida disminuye con la excentricidad visual y varía en función de la ubicación retinotópica. Además, se observa que el ancho de banda de sintonización depende de la excentricidad y está correlacionado con el pico de frecuencia espacial preferida.

En [\cite{kriegeskorte_understanding_2011}] se propone el desarrollo de modelos de campo receptivo en la corteza visual primaria utilizando fMRI. Critica los métodos convencionales como la \textit{medición de curvas de sintonización} y la \textit{clasificación de patrones multivariados}, y propone en su lugar la estimación del campo receptivo, método que permite medir respuestas a una amplia gama de estímulos y desarrollar modelos que describan cómo los estímulos se traducen en respuestas neuronales.

En [\cite{broderick_mapping_2022}] se desarrollaron modelos para analizar la sintonización de la frecuencia espacial en la corteza visual primaria humana. Se ajustaron curvas de sintonización log-normal a las respuestas de grupos de vóxeles en diferentes excentricidades, que permiten estimar la respuesta promedio BOLD\todo{ver si explico que es} en diferentes frecuencias espaciales. Los resultados muestran que la frecuencia espacial preferida varía inversamente con la excentricidad y se ve influenciada por la orientación del estímulo. Constituye una ampliaci\'on a la investigación previa sobre la sintonización de la frecuencia espacial en el córtex visual humano, como la realizada por [\cite{aghajari_population_2020}]. 






