\chapter{Discusi\'on}\label{chapter:discussion}

Se confirma que el tama\~no de pRF de \'areas corticales visuales crece con la excentricidad, y la pendiente de este crecimiento es mayor a medida que se avanza en direcci\'on posterior - anterios en la v\'ia visual. Se confirm\'o Adem\'as que el per\'iodo preferido de v\'ertices corticales en \'areas visuales crece con su excentricidad. Como hllazgo novedoso, se encontr\'o 

Encontramos una relaci\'on directa entre el pe\'riodo preferido y la excentricidad.

Encontramos que en cuatro \'areas visuales (hV4, VO1, TO2, V3b)  el per\'iodo preferido de los v\'ertices corticales era mayor en el hemisferio derecho que en el hemisferio izquierdo. El hemisferio derecho prefiere frecuencias espaciales m\'as bajas que el hemisferio izquierdo. Este efecto no apareci\'o en \'areas visuales primarias (V1-V3) y algunas \'areas de orden superior (ej. LO2).

Tambi\'en se demostr\'o que el tamano de \'areas visuales de los pRF es mayor en el hemisferio derecho 

Esto sugiere que la representaci\'on de las imagenes visuales en algunas \'areas corticales difiere en \'areas superiores.

Muy pocos estudios han examinado la diferencia entre campos receptivos del hemisferio izq-der. Los pocos que lo han hecho se han centrado en \'areas primarias y al igual que nosotros no encuentran diferencias. Los pocos trabajos que han examinado frecuencias espaciales no han comparado hemisferios. Este trabjo es el primer trabajo conocido que halla analiado la diferencia de frecuencias espaciales en ambos hemisferios.

Las estimaciones de pRF parecen m\'as ruidosas que las de preferiencia por sf. Esto se ve en en R2 de las dos variables. Lage y Castellano ha discutido las dificultades que existen en estimar con dificultad el tamono de pRF. 

Los resultados son consistentes con los datos neurosicologicoa neurofisiologico sy neurificos sobre lateral hemis, que coinciden en una ventaja del hemisferio derecho para lo global (sf)y de lo local para hem.izq  esta idea se postulo en la idea de doble filtraje de Robertson sin expecificar posibles circuitos neurales, nuestro analisis sugiere  que este mecanismo podr\'ia surgir de las propiedades discrepantes de los campos receptivos de los dos hemisferios. 

Simulaciones con redes neurales convolucionales prifundas realizadas en CNEURO, que fuereon entrenadas para reconocer letras, (Garea) apoyan esta idea, estas redes se contruyeron con multiples canales que diferian en el tamano y la selectividad en frecuencia de filtros de Gabor como primer paso del procesamiento de imagenes. Los filtros de gabor son el mejor modelo para describir el campo receptivo de celulas visuales o poblaciones neuronales. Cuando se entrena esta redes para discriminar objetos que tienen niveles globales y locales, eliminar los filtros de baja frecuencia impide la percepci\'on de letras locales. Seria interesante construir redes incorporando los parametros medidos en este estudio y hacer una sim mas realista.

La teoria del doble filtraje presupone un ajuste del rango de frecuencias en que se trabaja segun el tipo de estimulo al que entrenta el sujeto. Los sujetos miraban estimulos pequenos centrados en la fovea, es necesario cambiar esto para ver si la preferencia de sf varia. Se sabe que las propiedades de los campos receptivos cambian de acuerdo a la atencion

El estudio ofrece la primera propuesta de un modelo para explicar especializacion hemi de la percepcion glo/local y la primeras clave para la implementacion de RNN


