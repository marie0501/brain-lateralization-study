\begin{conclusions}	
	
	En esta investigación, se empleó un modelo lineal mixto para examinar la especialización hemisférica del cerebro en el procesamiento de frecuencias espaciales en la percepción visual.
	
	Los resultados revelaron que tanto el tamaño de los pRFs como el período preferido (inverso de las frecuencias espaciales) de los vértices corticales incrementan con la excentricidad en el campo visual, siendo mayor en \'areas visuales superiores (ej. TO1). A través del análisis de los resultados obtenidos, con el modelo lineal mixto, se observ\'o que el período preferido  de los vóxeles varía entre los hemisferios cerebrales en diversas áreas visuales, lo cual respalda la hipótesis de una lateralización hemisférica. Además, se not\'o que, en las áreas visuales donde se evidencia esta diferencia, el período preferido es más grande en el hemisferio derecho, indicando una inclinación de este hemisferio hacia frecuencias espaciales bajas. Estos hallazgos están en consonancia con datos neurofisiológicos y neuropsicológicos previos [\cite{flevaris_spatial_2016}] sobre la lateralización hemisférica.
	
	Por tanto, se concluye que los resultados de este estudio constituyen aportes significativos al campo de las neurociencias, especialmente en lo que respecta a la comprensión de la selectividad de los hemisferios cerebrales a diferentes frecuencias espaciales en la percepción visual. Los  hallazgos encontrados no solo profundizan nuestro conocimiento del procesamiento visual, sino que también destacan la complejidad y adaptabilidad del cerebro en la interpretación del entorno visual.
	
         
         
\end{conclusions}
