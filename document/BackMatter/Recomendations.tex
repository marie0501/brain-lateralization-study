\begin{recomendations}
    El estudio presenta recomendaciones cruciales para impulsar la investigación en especialización hemisférica del cerebro y su procesamiento de frecuencias espaciales en est\'imulos visuales.  
     \begin{itemize}
     	
     	\item [1.] Se propone revisar los datos actuales utilizando diferentes atlas retinotópicos, como el de Sereno [\cite{sereno_topological_2022}]. Esta estrategia permitiría identificar un mayor número de áreas visuales superiores, incluidas las que procesan caras, ofreciendo una visión más completa de la organización retinotópica en la corteza visual.
     	
     	\item [2.] Otra recomendación es desarrollar simulaciones con redes neuronales convolucionales que incorporen los parámetros fisiológicos de este estudio. Estas simulaciones podrían proporcionar una representación más exacta de cómo las frecuencias espaciales  influyen en la percepción visual, mejorando la comprensión de los procesos visuales subyacentes.
     	
     	\item[3.] Se sugiere aplicar la metodología de investigación a un conjunto de datos más amplio para obtener resultados más sólidos y confiables. Esta ampliación mejoraría la generalización de las conclusiones del estudio, fortaleciendo su validez.
     	
     	\item[4.] Finalmente, se recomienda realizar experimentos de fMRI enfocados en medir la preferencia por distintas frecuencias espaciales en ambos hemisferios teniendo en cuenta la variación en el espectro de los estímulos visuales con la atenci\'on. 
     	
     \end{itemize}
   
    
    
    
    
    
   
\end{recomendations}
