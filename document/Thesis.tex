\documentclass[12pt,oneside]{uhthesis}
\usepackage{subfigure}
\usepackage[ruled,lined,linesnumbered,titlenumbered,algochapter,spanish,onelanguage]{algorithm2e}
\usepackage{amsmath}
\usepackage{amssymb}
\usepackage{amsbsy}
\usepackage{caption,booktabs}
\captionsetup{ justification = centering }
%\usepackage{mathpazo}
\usepackage{float}
\setlength{\marginparwidth}{2cm}
\usepackage{todonotes}
\usepackage{listings}
\usepackage{xcolor}
\usepackage{multicol}
\usepackage{graphicx}
\floatstyle{plaintop}
\restylefloat{table}
\addbibresource{Bibliography.bib}
% \setlength{\parskip}{\baselineskip}%
\renewcommand{\tablename}{Tabla}
\renewcommand{\listalgorithmcfname}{Índice de Algoritmos}
%\dontprintsemicolon
\SetAlgoNoEnd

\definecolor{codegreen}{rgb}{0,0.6,0}
\definecolor{codegray}{rgb}{0.5,0.5,0.5}
\definecolor{codepurple}{rgb}{0.58,0,0.82}
\definecolor{backcolour}{rgb}{0.95,0.95,0.92}

\lstdefinestyle{mystyle}{
    backgroundcolor=\color{backcolour},   
    commentstyle=\color{codegreen},
    keywordstyle=\color{purple},
    numberstyle=\tiny\color{codegray},
    stringstyle=\color{codepurple},
    basicstyle=\ttfamily\footnotesize,
    breakatwhitespace=false,         
    breaklines=true,                 
    captionpos=b,                    
    keepspaces=true,                 
    numbers=left,                    
    numbersep=5pt,                  
    showspaces=false,                
    showstringspaces=false,
    showtabs=false,                  
    tabsize=4
}

\lstset{style=mystyle}

\title{Título de la tesis}
\author{\\\vspace{0.25cm}Nombre del autor}
\advisor{\\\vspace{0.25cm}Nombre del primer tutor\\\vspace{0.2cm}Nombre del segundo tutor}
\degree{Licenciado en (Matemática o Ciencia de la Computación)}
\faculty{Facultad de Matemática y Computación}
\date{Fecha\\\vspace{0.25cm}\href{https://github.com/username/repo}{github.com/username/repo}}
\logo{Graphics/uhlogo}
\makenomenclature

\renewcommand{\vec}[1]{\boldsymbol{#1}}
\newcommand{\diff}[1]{\ensuremath{\mathrm{d}#1}}
\newcommand{\me}[1]{\mathrm{e}^{#1}}
\newcommand{\pf}{\mathfrak{p}}
\newcommand{\qf}{\mathfrak{q}}
%\newcommand{\kf}{\mathfrak{k}}
\newcommand{\kt}{\mathtt{k}}
\newcommand{\mf}{\mathfrak{m}}
\newcommand{\hf}{\mathfrak{h}}
\newcommand{\fac}{\mathrm{fac}}
\newcommand{\maxx}[1]{\max\left\{ #1 \right\} }
\newcommand{\minn}[1]{\min\left\{ #1 \right\} }
\newcommand{\lldpcf}{1.25}
\newcommand{\nnorm}[1]{\left\lvert #1 \right\rvert }
\renewcommand{\lstlistingname}{Ejemplo de código}
\renewcommand{\lstlistlistingname}{Ejemplos de código}

\begin{document}

\frontmatter
\maketitle

\begin{dedication}
    \textbf{\textit{A mi familia, mi br\'ujula en la traves\'ia\\ y mi refugio en la tormenta.}}
\end{dedication}
\begin{acknowledgements}
    Agradecimientos
\end{acknowledgements}
\begin{opinion}
    La división de funciones entre los dos hemisferios cerebrales ha sido uno de los temas más intensamente estudiados en las neurociencias desde hace dos siglos, en particular en los procesos visuales. Esto ha generado un volumen valioso de datos psicológicos y clínicos. Con el desarrollo de los métodos modernos de neuroimágenes, se ha obtenido aún más información. Sin embargo, no han generado modelos cuantitativos que puedan explicar cómo se diferencian los dos hemisferios. Hay más datos que teoría sólida. El lenguaje natural para formular modelos apropiados parece estar en el terreno de redes neurales convolucionales profundas, las cuales deben su reciente éxito práctico en diversos campos tecnológicos en su imitación al sistema visual humano. El desarrollo de la modelación de los campos receptivos poblacionales a partir de datos de resonancia magnética funcional abre el camino a modelos de redes neurales convolucionales m\'as realistas.  La tesis de Mari\'e del Valle Reyes sería un primer paso en examinar con campos receptivos poblacionales la lateralización hemisférica de los procesos visuales.
    
    Mari\'e del Valle Reyes es una alumna muy destacada, trabajando con entusiasmo de forma incansable. La estudiante ha demostrado una integración valiosa de habilidades en matemáticas aplicadas a la neurociencia en su tesis de diploma. Ha combinado una capacidad de programación en varios lenguajes (como Python, R, y Matlab) con el aprendizaje rápido de modelos estadísticos avanzados, y el aprendizaje de temas de neurociencias. Su capacidad para traducir conceptos matemáticos complejos en herramientas aplicables a la comprensión de procesos neurales muestra un enfoque interdisciplinario y original. Se han obtenido resultados importantes ya casi maduros para publicar en poco tiempo gracias a su tenacidad y entrega al trabajo. Su investigación es una contribución prometedora en la convergencia de las matemáticas y las neurociencias. Ha sido un gusto trabajar con ella.
\end{opinion}
\begin{resumen}
	El estudio analiza cómo los hemisferios cerebrales se especializan en procesar frecuencias espaciales de est\'imulos visuales, mediante modelos de campos receptivos poblacionales y modelos lineales mixtos. Se enfoca en el período preferido de los vértices corticales en diferentes áreas visuales, explorando su relación con la excentricidad visual y su influencia en la lateralización hemisférica. Los resultados revelan que el hemisferio derecho tiende a procesar frecuencias más bajas que el hemisferio izquierdo en áreas visuales superiores, aunque esta tendencia no se observa en áreas visuales primarias.  Este trabajo favorece la comprensión de cómo se procesa y representa la información visual, contribuyendo significativamente al conocimiento sobre la especialización funcional de los hemisferios en distintos niveles de procesamiento visual.
\end{resumen}

\begin{abstract}
	The study examines how the cerebral hemispheres specialize in processing spatial frequencies of visual stimuli using models of population receptive fields and mixed linear models. It focuses on the preferred period of cortical vertices in different visual areas, exploring its relationship with visual eccentricity and its influence on hemispheric lateralization. The results reveal that the right hemisphere tends to process lower frequencies than the left hemisphere in higher visual areas, although this trend is not observed in primary visual areas. This work enhances our understanding of how visual information is processed and represented, making a significant contribution to our knowledge of the functional specialization of the hemispheres at different levels of visual processing.
\end{abstract}
\tableofcontents
\listoffigures
% \listoftables
% \listofalgorithms
\lstlistoflistings

\mainmatter

\chapter*{Introducción}\label{chapter:introduction}
\addcontentsline{toc}{chapter}{Introducción}
El cerebro humano se concibe como un sistema complejo que controla y regula la mayoría de las funciones del cuerpo y de la mente. Este \'organo desempeña un papel esencial en la percepción y el procesamiento de la información, debido a que es el encargado de recibir, interpretar y responder a los estímulos del entorno, lo cual permite a los individuos interactuar de manera efectiva con su mundo. La percepción implica la interpretación y organización de los estímulos sensoriales para formar una representación consciente de la realidad. El cerebro procesa la información visual, auditiva, táctil y otras modalidades sensoriales, integrándola para construir una experiencia coherente y significativa del entorno circundante.

Los dos hemisferios cerebrales desempeñan papeles notablemente diferentes en la percepción visual a pesar de su estrecha interacción, evidenciándose una lateralización hemisférica. Esta asimetría funcional se ha documentado exhaustivamente para procesar aspectos globales y locales de estímulos visuales \cite{flevaris_spatial_2016}, donde el hemisferio derecho muestra un sesgo global y el hemisferio izquierdo muestra una preferencia local. Estos sesgos para los niveles local/global se han establecido a través de métodos psicofísicos \cite{brederoo_hemispheric_2017, brederoo_reproducibility_2019} y electrofisiológicos \cite{flevaris_attending_2014, iglesias-fuster_asynchronous_2015, jiang_neural_2005}.

El sesgo global/local de los hemisferios derecho/izquierdo podría explicarse en términos de las frecuencias espaciales (SF) asociadas con diferentes niveles de un estímulo visual. Las frecuencias espaciales describen cómo varía la información visual en términos de patrones de luz y sombra en una escena. Estos patrones pueden involucrar detalles finos o cambios suaves en la luminosidad. En el contexto visual, las SF más bajas se asocian comúnmente con aspectos globales, mientras que las SF más altas se asocian con aspectos locales \cite{flevaris_spatial_2016}. En apoyo a esta idea, varios estudios han revelado un sesgo hacia SF más bajas en el hemisferio derecho y SF más altas en el hemisferio izquierdo \cite{flevaris_spatial_2016}. Esto implica que el hemisferio derecho puede ser más eficiente en procesar información visual relacionada con la configuración global de un estímulo, mientras que el hemisferio izquierdo podría destacarse en detalles locales más finos. Además, se han encontrado vínculos entre la selección atencional global/local y las frecuencias espaciales de los est\'imulos \cite{flevaris_spatial_2016}. Sin embargo, cualquier teoría basada en estas consideraciones debe tener en cuenta el hecho de que los mismos aspectos de un estímulo visual pueden ser globales en un contexto y locales en otro (por ejemplo, un árbol es global en relación con una hoja pero local en relación con un bosque). El papel de cualquier SF no es fijo sino que depende del contexto.	

Una posible forma de evaluar y medir la lateralización hemisférica se presenta a través de la medición de los campos receptivos de la población (pRF) mediante resonancia magnética funcional (fMRI) \cite{dumoulin_population_2008, kay_principles_2018} . Los pRF constituyen modelos cuantitativos que describen la respuesta combinada de las neuronas dentro de un vóxel de fMRI (vértice cortical). Estos modelos suelen estimar la posición y el tamaño de la sección del campo visual que afecta a un vóxel específico \cite{wandell_computational_2015}. La medición de respuestas de frecuencia espacial (SF) en los pRF también se ha convertido en un enfoque relevante, especialmente en áreas visuales tempranas (área visual primaria, V2, V3), ofreciendo perspectivas adicionales sobre la sintonización de SF en relación con el tamaño del pRF. Investigaciones recientes \cite{aghajari_population_2020, broderick_mapping_2022} han demostrado que la sintonización de la frecuencia espacial de un pRF tiende a disminuir a medida que aumenta su tamaño, lo que sugiere una posible relación entre la lateralización hemisférica y las características de procesamiento de la información visual en el cerebro. Estas observaciones brindan una valiosa perspectiva para comprender cómo la lateralización hemisférica puede estar vinculada a las propiedades de los campos receptivos de la población.

La investigación sobre las propiedades de los pRF en funci\'on del campo visual, se ha focalizado principalmente en las diferencias entre los cuadrantes superior e inferior en la corteza visual primaria (V1), donde no se han observado diferencias significativas entre los cuadrantes derecho e izquierdo. Un estudio encontró tamaños de pRF ligeramente más pequeños en el cuadrante izquierdo en comparación con los cuadrantes horizontales derechos de las \'areas visuales V2 y V3 y, nuevamente, no hubo diferencias en V1 \cite{silva_radial_2018}. Esta observación sugiere que la lateralidad de las propiedades de los pRF en áreas visuales intermedias y superiores no se ha estudiado exhaustivamente. La dificultad para identificar sitios homólogos entre hemisferios en estas regiones, donde los mapas retinotópicos son menos definidos, puede contribuir a la falta de investigación detallada en estas áreas.

Existen varias bases de datos p\'ublicas de pRF \cite{benson_bayesian_2018,himmelberg_cross-dataset_2021}, que cubren amplias extensiones de la corteza cerebral, lo que hace posible la prueba de la lateralidad de las propiedades de pRF en \'areas visuales de orden intermedio o superior. Para examinar las diferencias derecha/izquierda en el tamaño del pRF, se pueden utilizar dos estrategias denominadas aquí \textbf{anatómicas} y \textbf{homotópicas}. El objetivo es comparar sitios corticales homólogos, pero como se mencionó anteriormente, definir "homólogo" presenta dificultades, especialmente para áreas de orden superior con respuesta visual.

La estrategia anatómica mide las diferencias en las propiedades de pRF en un espacio donde las superficies corticales son aproximadamente simétricas en los hemisferios izquierdo y derecho. Esta simetría significa que cuando se considera el orden de los vértices, aquellos con el mismo rango en ambos hemisferios son aproximadamente homólogos.

El enfoque homotópico compara los regiones corticales con la conectividad anatómica o funcional más fuerte entre los dos hemisferios, una conexión que indica que probablemente trabajen juntos. Por lo tanto, la lateralización de las propiedades de pRF se puede examinar con una parcelación basada en la conectividad de resonancia magnética funcional en estado de reposo o relacionada con la tarea que identifica pares de áreas corticales homotópicas.  

El objetivo general de este estudio es determinar de manera sistemática si existen diferencias en las propiedades de los campos receptivos entre los hemisferios izquierdo y derecho en áreas visuales de orden intermedio y superior. Para alcanzar este propósito, se emplearán algoritmos computacionales especializados y pruebas estadísticas rigurosas. El análisis se llevará a cabo mediante la aplicación de estrategias previamente descritas a tres bases de datos pRF. La utilización de algoritmos computacionales permitirá la extracción y comparación de las propiedades de los campos receptivos en ambos hemisferios, mientras que la aplicación de pruebas estadísticas proporcionará una evaluación cuantitativa de la significancia de las posibles diferencias identificadas.

Para lograr el objetivo general del presente trabajo se
trazan los siguientes objetivos específicos:

\begin{itemize}
	\item Estudio del estado del arte sobre el preprocesamiento de las resonancias magn\'eticas funcionales.
	\item Estudiar el estado del arte sobre los modelos de campos receptivos poblacionales.
	\item Estudiar los elementos te\'oricos de la lateralidad hemisf\'erica cerebral.
	\item Implementar y evaluar las estrategias concebidas para la examinaci\'on de las diferencias en ambos hemisferios cerebrales en el tama\~no de los pRF.
	
\end{itemize}

En lo siguiente, esta tesis se divide en cuatro capítulos.  En el Capítulo 2, titulado "Marco Teórico-Conceptual", se realiza un análisis detallado del estado actual de la ciencia y tecnología en las áreas relevantes de la esta investigaci\'on. En el Capítulo 3, titulado ''Concepción y Diseño de las Estrategias", se describe en detalle la metodología para abordar la investigación sobre la lateralidad hemisférica, incluyendo aspectos clave del enfoque analítico. Detalles técnicos de la implementación del sistema se presentan en el Capítulo 4, titulado ''Implementación y Experimentación". En este capítulo, se explora cualitativamente la validez de la solución implementada, aprovechando las herramientas disponibles. Se describen los métodos y técnicas utilizadas para evaluar la lateralidad hemisférica en áreas visuales de orden intermedio y superior, destacando los resultados y observaciones obtenidas durante la experimentación. En la parte del desenlace, se presenta el Capítulo 5, donde se exponen las conclusiones de la investigación. Se destacan los logros clave en relación con los objetivos planteados, proporcionando un resumen de los hallazgos más significativos. Además, se presentan recomendaciones que señalan futuras direcciones de investigación, brindando perspectivas para la continuidad del estudio sobre la lateralidad hemisférica y el procesamiento visual. Finalmente, la bibliografía utilizada para respaldar la base científica de la solución propuesta y los anexos complementarios se incluyen en secciones respectivas, facilitando la exploración de temas relacionados y proporcionando una base sólida para la validación y respaldo de la investigación realizada.

\chapter{Estado del Arte}\label{chapter:state-of-the-art}

\section*{Lateralidad Hemisf\'erica}
\section*{Mapas Retinot\'opicos}
\section*{pRF}

\chapter{Propuesta}\label{chapter:proposal}

\chapter{Detalles de Implementación y Experimentos}\label{chapter:implementation}


\backmatter

\begin{conclusions}	
	
	En esta investigación, se empleó un modelo lineal mixto para examinar la especialización hemisférica del cerebro en el procesamiento de frecuencias espaciales en la percepción visual.
	
	Los resultados revelaron que tanto el tamaño de los pRFs como el período preferido (inverso de las frecuencias espaciales) de los vértices corticales incrementan con la excentricidad en el campo visual, siendo mayor en \'areas visuales superiores (ej. TO1). A través del análisis de los resultados obtenidos, con el modelo lineal mixto, se observ\'o que el período preferido  de los vóxeles varía entre los hemisferios cerebrales en diversas áreas visuales, lo cual respalda la hipótesis de una lateralización hemisférica. Además, se not\'o que, en las áreas visuales donde se evidencia esta diferencia, el período preferido es más grande en el hemisferio derecho, indicando una inclinación de este hemisferio hacia frecuencias espaciales bajas. Estos hallazgos están en consonancia con datos neurofisiológicos y neuropsicológicos previos [\cite{flevaris_spatial_2016}] sobre la lateralización hemisférica.
	
	Por tanto, se concluye que los resultados de este estudio constituyen aportes significativos al campo de las neurociencias, especialmente en lo que respecta a la comprensión de la selectividad de los hemisferios cerebrales a diferentes frecuencias espaciales en la percepción visual. Los  hallazgos encontrados no solo profundizan nuestro conocimiento del procesamiento visual, sino que también destacan la complejidad y adaptabilidad del cerebro en la interpretación del entorno visual.
	
         
         
\end{conclusions}

\begin{recomendations}
    Recomendaciones
\end{recomendations}

\printbibliography[heading=bibintoc]



\end{document}