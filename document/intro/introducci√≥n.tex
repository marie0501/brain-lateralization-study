\documentclass{article}
\usepackage[spanish]{babel}

\title{}
\author{Mari\'e del Valle Reyes}
\date{C411}

\begin{document}
	
	\maketitle
	
	\section*{Introducci\'on}
	%---------------------
	El cerebro humano se concibe como un sistema complejo que controla y regula la mayoría de las funciones del cuerpo y de la mente. Este \'organo desempeña un papel esencial en la percepción y el procesamiento de la información, debido a que es el encargado de recibir, interpretar y responder a los estímulos del entorno, lo cual permite a los individuos interactuar de manera efectiva con su mundo. La percepción implica la interpretación y organización de los estímulos sensoriales para formar una representación consciente de la realidad. El cerebro procesa la información visual, auditiva, táctil y otras modalidades sensoriales, integrándola para construir una experiencia coherente y significativa del entorno circundante.
	
	Los dos hemisferios cerebrales desempeñan papeles notablemente diferentes en la percepción visual a pesar de su estrecha interacción, evidenciándose una lateralización hemisférica. Esta asimetría funcional se ha documentado exhaustivamente para procesar aspectos globales y locales de estímulos visuales \cite{flevaris_spatial_2016}, donde el hemisferio derecho muestra un sesgo global y el hemisferio izquierdo muestra una preferencia local. Estos sesgos para los niveles local/global se han establecido a través de métodos psicofísicos \cite{brederoo_hemispheric_2017, brederoo_reproducibility_2019} y electrofisiológicos \cite{flevaris_attending_2014, iglesias-fuster_asynchronous_2015, jiang_neural_2005}.
	
	El sesgo global/local de los hemisferios derecho/izquierdo podría explicarse en términos de las frecuencias espaciales (SF) asociadas con diferentes niveles de un estímulo visual. Las frecuencias espaciales describen cómo varía la información visual en términos de patrones de luz y sombra en una escena. Estos patrones pueden involucrar detalles finos o cambios suaves en la luminosidad. En el contexto visual, las SF más bajas se asocian comúnmente con aspectos globales, mientras que las SF más altas se asocian con aspectos locales \cite{flevaris_spatial_2016}. En apoyo a esta idea, varios estudios han revelado un sesgo hacia SF más bajas en el hemisferio derecho y SF más altas en el hemisferio izquierdo \cite{flevaris_spatial_2016}. Esto implica que el hemisferio derecho puede ser más eficiente en procesar información visual relacionada con la configuración global de un estímulo, mientras que el hemisferio izquierdo podría destacarse en detalles locales más finos. Además, se han encontrado vínculos entre la selección atencional global/local y las frecuencias espaciales de los est\'imulos \cite{flevaris_spatial_2016}. Sin embargo, cualquier teoría basada en estas consideraciones debe tener en cuenta el hecho de que los mismos aspectos de un estímulo visual pueden ser globales en un contexto y locales en otro (por ejemplo, un árbol es global en relación con una hoja pero local en relación con un bosque). El papel de cualquier SF no es fijo sino que depende del contexto.	
	
	Una posible forma de evaluar y medir la lateralización hemisférica se presenta a través de la medición de los campos receptivos de la población (pRF) mediante resonancia magnética funcional (fMRI) \cite{dumoulin_population_2008, kay_principles_2018} . Los pRF constituyen modelos cuantitativos que describen la respuesta combinada de las neuronas dentro de un vóxel de fMRI (vértice cortical). Estos modelos suelen estimar la posición y el tamaño de la sección del campo visual que afecta a un vóxel específico \cite{wandell_computational_2015}. La medición de respuestas de frecuencia espacial (SF) en los pRF también se ha convertido en un enfoque relevante, especialmente en áreas visuales tempranas (área visual primaria, V2, V3), ofreciendo perspectivas adicionales sobre la sintonización de SF en relación con el tamaño del pRF. Investigaciones recientes \cite{aghajari_population_2020, broderick_mapping_2022} han demostrado que la sintonización de la frecuencia espacial de un pRF tiende a disminuir a medida que aumenta su tamaño, lo que sugiere una posible relación entre la lateralización hemisférica y las características de procesamiento de la información visual en el cerebro. Estas observaciones brindan una valiosa perspectiva para comprender cómo la lateralización hemisférica puede estar vinculada a las propiedades de los campos receptivos de la población.
	
	La investigación sobre las propiedades de los pRF en funci\'on del campo visual, se ha focalizado principalmente en las diferencias entre los cuadrantes superior e inferior en la corteza visual primaria (V1), donde no se han observado diferencias significativas entre los cuadrantes derecho e izquierdo. Un estudio encontró tamaños de pRF ligeramente más pequeños en el cuadrante izquierdo en comparación con los cuadrantes horizontales derechos de las \'areas visuales V2 y V3 y, nuevamente, no hubo diferencias en V1 \cite{silva_radial_2018}. Esta observación sugiere que la lateralidad de las propiedades de los pRF en áreas visuales intermedias y superiores no se ha estudiado exhaustivamente. La dificultad para identificar sitios homólogos entre hemisferios en estas regiones, donde los mapas retinotópicos son menos definidos, puede contribuir a la falta de investigación detallada en estas áreas.
			
	Existen varias bases de datos p\'ublicas de pRF \cite{benson_bayesian_2018,himmelberg_cross-dataset_2021}, que cubren amplias extensiones de la corteza cerebral, lo que hace posible la prueba de la lateralidad de las propiedades de pRF en \'areas visuales de orden intermedio o superior. Para examinar las diferencias derecha/izquierda en el tamaño del pRF, se pueden utilizar dos estrategias denominadas aquí \textbf{anatómicas} y \textbf{homotópicas}. El objetivo es comparar sitios corticales homólogos, pero como se mencionó anteriormente, definir "homólogo" presenta dificultades, especialmente para áreas de orden superior con respuesta visual.
	
	La estrategia anatómica mide las diferencias en las propiedades de pRF en un espacio donde las superficies corticales son aproximadamente simétricas en los hemisferios izquierdo y derecho. Esta simetría significa que cuando se considera el orden de los vértices, aquellos con el mismo rango en ambos hemisferios son aproximadamente homólogos.
	
	 El enfoque homotópico compara los regiones corticales con la conectividad anatómica o funcional más fuerte entre los dos hemisferios, una conexión que indica que probablemente trabajen juntos. Por lo tanto, la lateralización de las propiedades de pRF se puede examinar con una parcelación basada en la conectividad de resonancia magnética funcional en estado de reposo o relacionada con la tarea que identifica pares de áreas corticales homotópicas.  
	
	El objetivo general de este estudio es determinar de manera sistemática si existen diferencias en las propiedades de los campos receptivos entre los hemisferios izquierdo y derecho en áreas visuales de orden intermedio y superior. Para alcanzar este propósito, se emplearán algoritmos computacionales especializados y pruebas estadísticas rigurosas. El análisis se llevará a cabo mediante la aplicación de estrategias previamente descritas a tres bases de datos pRF. La utilización de algoritmos computacionales permitirá la extracción y comparación de las propiedades de los campos receptivos en ambos hemisferios, mientras que la aplicación de pruebas estadísticas proporcionará una evaluación cuantitativa de la significancia de las posibles diferencias identificadas.
	
	Para lograr el objetivo general del presente trabajo se
	trazan los siguientes objetivos específicos:
	
	\begin{itemize}
		\item Estudio del estado del arte sobre el preprocesamiento de las resonancias magn\'eticas funcionales.
		\item Estudiar el estado del arte sobre los modelos de campos receptivos poblacionales.
		\item Estudiar los elementos te\'oricos de la lateralidad hemisf\'erica cerebral.
		\item Implementar y evaluar las estrategias concebidas para la examinaci\'on de las diferencias en ambos hemisferios cerebrales en el tama\~no de los pRF.
	
	\end{itemize}

En lo siguiente, esta tesis se divide en cuatro capítulos.  En el Capítulo 2, titulado "Marco Teórico-Conceptual", se realiza un análisis detallado del estado actual de la ciencia y tecnología en las áreas relevantes de la esta investigaci\'on. En el Capítulo 3, titulado ''Concepción y Diseño de las Estrategias", se describe en detalle la metodología para abordar la investigación sobre la lateralidad hemisférica, incluyendo aspectos clave del enfoque analítico. Detalles técnicos de la implementación del sistema se presentan en el Capítulo 4, titulado ''Implementación y Experimentación". En este capítulo, se explora cualitativamente la validez de la solución implementada, aprovechando las herramientas disponibles. Se describen los métodos y técnicas utilizadas para evaluar la lateralidad hemisférica en áreas visuales de orden intermedio y superior, destacando los resultados y observaciones obtenidas durante la experimentación. En la parte del desenlace, se presenta el Capítulo 5, donde se exponen las conclusiones de la investigación. Se destacan los logros clave en relación con los objetivos planteados, proporcionando un resumen de los hallazgos más significativos. Además, se presentan recomendaciones que señalan futuras direcciones de investigación, brindando perspectivas para la continuidad del estudio sobre la lateralidad hemisférica y el procesamiento visual. Finalmente, la bibliografía utilizada para respaldar la base científica de la solución propuesta y los anexos complementarios se incluyen en secciones respectivas, facilitando la exploración de temas relacionados y proporcionando una base sólida para la validación y respaldo de la investigación realizada.
	
	%---------------------
	\newpage
	\bibliography{ref}
	\bibliographystyle{apalike}
	
	\newpage
	
	Durante la segunda mitad del siglo XX, el foco central de la biología estuvo en el gen. Ahora, en la primera mitad del siglo XXI, la atención se ha desplazado hacia la ciencia neuronal y, específicamente, hacia la biología de la mente. Deseamos comprender los procesos mediante los cuales percibimos, actuamos, aprendemos y recordamos.
	La visión, un sentido particularmente bien comprendido y ampliamente utilizado por los humanos, adquiere información a través de las propiedades de la luz. La luz reflejada por los objetos del entorno varía en longitud de onda e intensidad y fluctúa en el espacio y el tiempo y, a través de esas propiedades físicas, transmite evidencia del mundo que nos rodea. Proyectada como imágenes estampadas sobre la retina, la energía luminosa se transduce en señales neuronales mediante células receptoras específicas. Las propiedades probatorias de estas imágenes son detectadas por un conjunto de sistemas neuronales especializados que detectan formas de contraste y transmiten esta información al resto del cerebro.
	
	
	El cerebro humano es el órgano central del sistema nervioso, una estructura sumamente especializada que procesa información, regula funciones corporales y, de manera crucial, da origen a nuestra cognición. Dentro de este órgano se encuentra la clave para comprender el misterio de la mente humana. Las neuronas, células especializadas en la transmisión de señales eléctricas, forman intrincadas redes que constituyen la base de nuestros pensamientos, percepciones y experiencias.
	
	El estudio del cerebro ha sido una empresa fascinante a lo largo de la historia, impulsada por el deseo de comprender los misterios de la mente humana y los procesos cognitivos. A medida que avanzamos en el tiempo, la búsqueda de herramientas para desentrañar los secretos del cerebro ha experimentado notables evoluciones. Desde los rudimentarios métodos de observación hasta las sofisticadas tecnologías de imagenología cerebral de hoy en día, el progreso en esta área ha sido notable. En la actualidad, la resonancia magnética funcional (fMRI) se ha convertido en una herramienta indispensable en la investigación cerebral. Captura la actividad neuronal mediante la detección de cambios en el flujo sanguíneo, proporcionando imágenes detalladas de las regiones activas durante diversas tareas cognitivas. La capacidad de visualizar la actividad cerebral mientras los participantes realizan tareas cognitivas específicas ha permitido a los investigadores hacer avances significativos en la comprensión de la base neural de la cognición y el comportamiento.
	
	La percepción visual, esencial para nuestra comprensión del entorno, se ha convertido en un área central de estudio en la neurociencia cognitiva. Esta inicia en la vía visual primaria, sistema que desempeña un papel crucial al transmitir información visual desde los ojos hasta la corteza visual primaria (V1), ubicada en el lóbulo occipital del cerebro, donde se lleva a cabo un procesamiento más avanzado que nos permite interpretar características visuales como color, forma y orientación.
	
	La lateralización hemisférica, un fenómeno fascinante del cerebro humano, ha capturado la atención de los investigadores y científicos cognitivos durante décadas. Este concepto se refiere a la asimetría funcional y estructural entre los hemisferios cerebrales derecho e izquierdo, destacando la notable especialización de cada uno en ciertas funciones cognitivas y procesos mentales. Entre las diversas facetas de la percepción visual, la lateralización hemisférica del cerebro ha emergido como un fenómeno intrigante, especialmente en relación con la percepción global y local de estímulos visuales.
	
	La lateralización hemisférica en la percepción visual se manifiesta en la asimetría funcional entre los dos hemisferios cerebrales, donde el hemisferio derecho tiende a especializarse en la percepción global, capturando la esencia general de un estímulo visual, mientras que el hemisferio izquierdo se inclina hacia la percepción local, centrándose en detalles específicos. Un elemento clave en esta dicotomía perceptual es la frecuencia espacial, que se refiere a la densidad de cambios en la luminancia de una imagen. Las frecuencias espaciales más bajas están asociadas con patrones más amplios y globales, mientras que las frecuencias más altas representan detalles más finos y locales. Esta relación entre la lateralización hemisférica y la percepción global/local se basa en cómo cada hemisferio procesa y prioriza las distintas frecuencias espaciales, contribuyendo a nuestra comprensión de la escena visual.
	
	En el fascinante viaje de explorar la lateralización hemisférica y su vínculo con la percepción visual, se hace imperativo contar con un marco conceptual sólido que guíe nuestra investigación. Aquí es donde los tres niveles propuestos por David Marr —computacional, algorítmico e implementación— se revelan como herramientas fundamentales para desentrañar los misterios que yacen en la organización neuronal de la percepción visual.
	
	En la búsqueda constante por comprender los intrincados procesos que rigen la percepción visual y la lateralización hemisférica en el cerebro humano, las herramientas de investigación desempeñan un papel crucial. En este contexto, los Campos Receptivos Poblacionales (pRF) emergen como una herramienta esencial para desvelar los secretos de la activación cerebral en cada hemisferio.
	
	Los pRF, o campos receptivos poblacionales, representan una ventana única a la actividad neuronal, permitiendo la cuantificación de la respuesta neuronal a estímulos visuales específicos. Estos campos receptivos no solo capturan la respuesta de neuronas individuales, sino que, de manera ingeniosa, consideran la población de neuronas que responde a un estímulo particular. Así, los pRF se convierten en una herramienta poderosa para mapear la organización espacial de la activación cerebral en respuesta a estímulos visuales.
	
	En términos más concretos, un pRF es la región del espacio visual que activa una unidad neuronal específica. Cuando se expande esta noción a la escala poblacional, obtenemos una representación detallada de cómo diferentes áreas cerebrales, y específicamente cada hemisferio, responden a estímulos visuales variados. La activación resultante se convierte en un marcador invaluable para entender cómo el cerebro procesa, organiza y extrae información visual.
	
	En esta investigación, aprovechamos la capacidad única de los pRF para proporcionar un mapeo detallado de la activación hemisférica en respuesta a estímulos visuales específicos. Al emplear esta herramienta, no solo aspiramos a visualizar la lateralización hemisférica en acción, sino también a identificar patrones distintivos que caracterizan la percepción visual en cada hemisferio. Los pRF se erigen así como una llave maestra para desbloquear las complejidades de la organización neural en el estudio de la percepción visual y la lateralización hemisférica.
	\newpage
	
	Claro, aquí está la introducción revisada sin el punto 4:
	
	1. **El Cerebro como Sistema Inteligente:**
	- Descripción general del cerebro como un sistema inteligente con estructura y función.
	- Ejemplos de funciones cerebrales, destacando el procesamiento de información como una función clave.
	
	2. **Asimetrías en la Función Cerebral:**
	- Reconocimiento de asimetrías significativas en la función cerebral.
	- Mención de las funciones específicas de los hemisferios izquierdo (HI) y derecho (HD).
	
	3. **Importancia de la Información Visual:**
	- Destacar la relevancia de la información visual en el conjunto de funciones cerebrales.
	- Enfatizar la importancia de la percepción visual para la comprensión del entorno.
	
	5. **Desafíos de las Teorías Iniciales:**
	- Crítica a las teorías iniciales que buscaban una especialización simple de tareas para HI y HD.
	- Necesidad de un enfoque más integral para comprender las funciones cerebrales.
	
	6. **Teoría del Doble Filtro Frecuencial (DFF):**
	- Introducción y explicación de la teoría DFF como un marco conceptual.
	- Relación de la teoría con la sensibilidad hemisférica a información de baja y alta frecuencia espacial.
	
	7. **Uso de fMRI en la Captura de Actividad Neural:**
	- Breve descripción del uso de resonancia magnética funcional (fMRI) como herramienta para registrar la actividad cerebral.
	
	8. **Hipótesis de Lateralización Hemisférica:**
	- Presentación de la hipótesis de lateralización hemisférica en áreas superiores en seres humanos.
	- Justificación de la hipótesis y su relevancia para la investigación.
	
	9. **Utilización de Campos Receptivos Poblacionales (pRF):**
	- Explicación de cómo se utilizarán los pRF para evaluar la activación en ambos hemisferios.
	- Descripción de las variantes de comparación, incluyendo la comparación de vértices homólogos y áreas específicas.
	
	\newpage
	
	**Introducción**
	
	El cerebro humano, una maravilla de complejidad, sirve como el epicentro de nuestra cognición y experiencias. En su intrincada estructura, se teje la esencia de nuestra identidad, procesando información de manera inigualable y dando vida a nuestra comprensión del entorno. Desde la toma de decisiones hasta la formación de recuerdos, el cerebro despliega una red de funciones que definen nuestra existencia.
	
	Al explorar este órgano, nos enfrentamos a la fascinante realidad de las asimetrías significativas en su funcionamiento. En la mayoría, el hemisferio izquierdo (HI) lidera en la producción y percepción del lenguaje, mientras que el hemisferio derecho (HD) destaca en la atención a aspectos espaciales. Sin embargo, este panorama no puede reducirse a dicotomías simples, y es aquí donde surge la teoría del Doble Filtro Frecuencial (DFF). Esta teoría nos lleva más allá de las simplificaciones iniciales, proponiendo una comprensión más integral de cómo los hemisferios responden a diferentes frecuencias espaciales en la información visual.
	
	La percepción visual, fundamental para entender nuestro entorno, se convierte en un enigma particular. ¿Cómo procesa el cerebro la información visual, y cómo estas funciones se distribuyen entre los hemisferios? La resonancia magnética funcional (fMRI) emerge como una herramienta vital, permitiéndonos capturar la actividad neuronal mientras enfrentamos esta interrogante. Este viaje de exploración se extiende a la hipótesis de la lateralización hemisférica en áreas superiores de los seres humanos, un fenómeno que se despliega en la percepción global y local de estímulos visuales.
	
	En la búsqueda de respuestas, nos apoyamos en los Campos Receptivos Poblacionales (pRF), modelando cómo áreas cerebrales responden a estímulos visuales específicos. Estos pRF se convierten en la lente a través de la cual exploramos la activación en cada hemisferio. Al comparar vértices homólogos y áreas específicas, buscamos entender las diferencias y similitudes en la respuesta hemisférica, desentrañando así los secretos de la organización neural detrás de la percepción visual y la lateralización hemisférica.
	
	\newpage
	
	El cerebro humano, el órgano central del sistema nervioso, es una estructura altamente especializada que regula funciones corporales, procesa información y da origen a nuestra cognición. En este órgano, las neuronas forman intrincadas redes que constituyen la base de nuestros pensamientos, percepciones y experiencias.
	
	La historia del estudio del cerebro ha sido fascinante, impulsada por el deseo de comprender los misterios de la mente humana y los procesos cognitivos. Desde los primeros métodos de observación hasta las tecnologías actuales de imagenología cerebral, como la resonancia magnética funcional (fMRI), hemos avanzado notablemente. La fMRI se ha vuelto esencial, capturando la actividad neuronal y proporcionando imágenes detalladas de regiones activas durante tareas cognitivas.
	
	La percepción visual, esencial para entender nuestro entorno, se centra en la vía visual primaria. Esta vía transmite información desde los ojos hasta la corteza visual primaria (V1), donde se realiza un procesamiento avanzado de características visuales.
	
	La lateralización hemisférica, la asimetría funcional entre los hemisferios cerebrales derecho e izquierdo, es fascinante en la percepción visual. Se destaca en la percepción global/local, donde el hemisferio derecho se especializa en lo global y el izquierdo en lo local. La frecuencia espacial, relacionada con cambios en la luminancia, influye en esta dicotomía perceptual.
	
	En nuestro estudio, adoptamos los niveles de Marr para guiar la investigación sobre la organización neuronal. La lateralización hemisférica y la percepción visual se explorarán mediante Campos Receptivos Poblacionales (pRF), modelando la respuesta neuronal a estímulos. Esto proporcionará un mapeo detallado de la activación hemisférica, revelando patrones distintivos en cada hemisferio.
	
	Los pRF, al considerar la población neuronal, se convierten en una herramienta clave para entender la organización neural en la percepción visual y la lateralización hemisférica. Este enfoque promete desbloquear complejidades y avanzar en nuestra comprensión de la mente humana.
	
	-----
	
	En este contexto, la presente tesis se sumerge en el estudio de la lateralización hemisférica en la percepción visual, explorando cómo los hemisferios cerebrales procesan y interpretan información visual, particularmente en relación con la frecuencia espacial y la dicotomía global/local. Al utilizar la resonancia magnética funcional (fMRI), esta investigación busca desentrañar los mecanismos que subyacen a la percepción visual y ampliar nuestro entendimiento de cómo el cerebro organiza y da sentido a la riqueza visual que nos rodea.
	
	1. **Comprensión de la Identidad:** Investigar el funcionamiento cerebral nos permite desentrañar la compleja red de procesos que conforman nuestra identidad. Desde la toma de decisiones hasta la formación de recuerdos, cada aspecto de nuestra existencia tiene su correlato en las estructuras cerebrales.
	
	2. **Avances Médicos:** El estudio del cerebro impulsa avances en el campo médico, facilitando diagnósticos más precisos y tratamientos efectivos para trastornos neurológicos y psicológicos.
	
	3. **Desarrollo de Terapias:** Comprender cómo el cerebro responde a diferentes estímulos es fundamental para el desarrollo de terapias y tratamientos destinados a mejorar la salud mental y emocional.
	
	4. **Innovación Tecnológica:** La investigación cerebral inspira innovaciones tecnológicas, desde interfaces cerebro-máquina hasta tecnologías de asistencia que mejoran la calidad de vida.
	
	5. **Abordaje de Problemas Sociales:** Problemas sociales como la educación y la toma de decisiones pueden beneficiarse de un entendimiento más profundo de cómo el cerebro procesa la información.
	
	En este contexto, explorar el cerebro se convierte en un viaje fascinante hacia el corazón mismo de la condición humana. Esta investigación no solo arroja luz sobre el órgano más complejo de nuestro cuerpo, sino que también nos acerca a la comprensión de lo que significa ser consciente, pensar y sentir. Es a través del estudio del cerebro que desentrañamos los secretos de nuestra propia existencia.
	
	
	
	
	
	Este viaje a través de las herramientas de estudio del cerebro refleja la constante búsqueda de métodos más precisos y reveladores. A medida que exploramos estas herramientas a lo largo de las décadas, nos acercamos a comprender más completamente el órgano más complejo del cuerpo humano.
	
	Estudiar el cerebro implica, en última instancia, explorar los cimientos mismos de lo que significa ser humano.
	
	\begin{itemize}
		\item 1. Importancia del estudio del cerebro, en particular, su funcion teniendo en cuenta los procesos cognitivos y explicar brevemente en que consisten estos.
		\item 2. Mencionar y explicar metodos para registrar y ser capaces de estudiar la funcion del cerebro, desde su surgimiento hasta hoy en dia, y queen esta tesis se utilizan fMRI.
		\item 3. Estudios que se han realizado sobre el cerebro en cuanto a funcion como la especializacion funcional de diferentes areas cerebrales.
		\item 4. Explicar que esta tesis consiste en el estudio de la lateralizacion hemisferica y en que consiste esto.
		\item 5. Mencionar que voy a utilizar los niveles de Marr como guia en este estudio y explicar por que. 
		\item 6. Explicar utilizacion de campos receptivos poblacionales (pRF) como herramienta para medir la activacion en cada hemisferio, explicando tambien brevemente en que consisten. 
		\item 7.Importancia de este estudio.
		\item 8. Explicar que se utilizan de varias herramientas computacionales  en el preprocesamiento de las fMRI, en la estimacion de parametros de pRF y en la definicion de areas cerebrales y de test estadisticos para la evaluacion de la hipotesis.
	\end{itemize}
	
	\subsection{1}
	**Introducción: Explorando la Mente a Través del Cerebro**
	
	El cerebro humano, con su intrincada red de neuronas y complejas conexiones sinápticas, es el epicentro de la experiencia humana. Este órgano asombroso, contenido dentro del cráneo, no solo coordina nuestras funciones vitales, sino que también alberga la esencia misma de lo que somos: nuestra mente.
	
	**El Cerebro: Epicentro de la Experiencia Humana**
	
	El cerebro es el órgano central del sistema nervioso, una estructura sumamente especializada que procesa información, regula funciones corporales y, de manera crucial, da origen a nuestra cognición. Su complejidad radica en su capacidad para generar pensamientos, emociones, recuerdos y coordinar acciones, todo ello en un delicado equilibrio.
	
	Dentro de este órgano se encuentra la clave para comprender el misterio de la mente humana. Las neuronas, células especializadas en la transmisión de señales eléctricas, forman intrincadas redes que constituyen la base de nuestros pensamientos, percepciones y experiencias. Estudiar el cerebro implica, en última instancia, explorar los cimientos mismos de lo que significa ser humano.
	
	**Importancia del Estudio del Funcionamiento Cerebral:**
	
	1. **Comprensión de la Identidad:** Investigar el funcionamiento cerebral nos permite desentrañar la compleja red de procesos que conforman nuestra identidad. Desde la toma de decisiones hasta la formación de recuerdos, cada aspecto de nuestra existencia tiene su correlato en las estructuras cerebrales.
	
	2. **Avances Médicos:** El estudio del cerebro impulsa avances en el campo médico, facilitando diagnósticos más precisos y tratamientos efectivos para trastornos neurológicos y psicológicos.
	
	3. **Desarrollo de Terapias:** Comprender cómo el cerebro responde a diferentes estímulos es fundamental para el desarrollo de terapias y tratamientos destinados a mejorar la salud mental y emocional.
	
	4. **Innovación Tecnológica:** La investigación cerebral inspira innovaciones tecnológicas, desde interfaces cerebro-máquina hasta tecnologías de asistencia que mejoran la calidad de vida.
	
	5. **Abordaje de Problemas Sociales:** Problemas sociales como la educación y la toma de decisiones pueden beneficiarse de un entendimiento más profundo de cómo el cerebro procesa la información.
	
	La investigación del funcionamiento cerebral desentraña la compleja red de procesos que constituyen nuestra identidad, abarcando desde la toma de decisiones hasta la formación de recuerdos. Cada aspecto de nuestra existencia tiene su correlato en las intrincadas estructuras cerebrales, lo que promueve una comprensión más profunda de nuestra identidad y comportamiento.
	
	Este conocimiento no solo tiene implicaciones a nivel individual, sino que también impulsa avances médicos significativos. Facilita diagnósticos más precisos y contribuye al desarrollo de tratamientos efectivos para trastornos neurológicos y psicológicos, marcando una pauta importante en el campo de la salud.
	
	Además, la comprensión de cómo el cerebro responde a diferentes estímulos es esencial para el desarrollo de terapias y tratamientos destinados a mejorar la salud mental y emocional. Este enfoque no solo se traduce en avances médicos, sino que también impacta directamente en la calidad de vida de las personas.
	
	La investigación cerebral no solo se limita a los beneficios médicos, sino que también inspira innovaciones tecnológicas. Desde interfaces cerebro-máquina hasta tecnologías de asistencia, estos avances no solo amplían nuestro conocimiento, sino que también tienen aplicaciones prácticas que mejoran la vida cotidiana.
	
	Además, este conocimiento profundo del cerebro también contribuye al abordaje de problemas sociales. Desde mejorar la educación hasta entender la toma de decisiones, la investigación cerebral proporciona herramientas valiosas para enfrentar y resolver problemas que afectan a la sociedad en su conjunto. En resumen, el estudio del cerebro no solo amplía nuestro entendimiento de la mente humana, sino que también tiene impactos tangibles en la medicina, la tecnología y la sociedad en su conjunto.
	
	En este contexto, explorar el cerebro se convierte en un viaje fascinante hacia el corazón mismo de la condición humana. Esta investigación no solo arroja luz sobre el órgano más complejo de nuestro cuerpo, sino que también nos acerca a la comprensión de lo que significa ser consciente, pensar y sentir. Es a través del estudio del cerebro que desentrañamos los secretos de nuestra propia existencia.
	
	\subsection{2}
	
	El estudio del cerebro ha sido una empresa fascinante a lo largo de la historia, impulsada por el deseo de comprender los misterios de la mente humana y los procesos cognitivos. A medida que avanzamos en el tiempo, la búsqueda de herramientas para desentrañar los secretos del cerebro ha experimentado notables evoluciones. Desde los rudimentarios métodos de observación hasta las sofisticadas tecnologías de imagenología cerebral de hoy en día, el progreso en esta área ha sido notable.
	
	En los albores de la investigación cerebral, los anatomistas observaban con atención las estructuras cerebrales post mortem, construyendo gradualmente un conocimiento anatómico. Sin embargo, la comprensión funcional del cerebro requería métodos más dinámicos. La llegada de la electroencefalografía (EEG) en el siglo XX permitió registrar la actividad eléctrica cerebral en tiempo real, marcando un hito en la investigación neurofisiológica.
	
	El desarrollo de la tomografía computarizada (TC) en la década de 1970 y la resonancia magnética (RM) en la década de 1980 proporcionaron imágenes más detalladas del cerebro en vivo. Estas técnicas no solo mejoraron la resolución espacial, sino que también permitieron explorar la estructura y la función simultáneamente.
	
	Hoy en día, la resonancia magnética funcional (fMRI) se ha convertido en una herramienta indispensable en la investigación cerebral. Captura la actividad neuronal mediante la detección de cambios en el flujo sanguíneo, proporcionando imágenes detalladas de las regiones activas durante diversas tareas cognitivas.
	
	Este viaje a través de las herramientas de estudio del cerebro refleja la constante búsqueda de métodos más precisos y reveladores. A medida que exploramos estas herramientas a lo largo de las décadas, nos acercamos a comprender más completamente el órgano más complejo del cuerpo humano.
	
	\subsection{3}
	Introducción:
	
	La investigación del cerebro mediante resonancia magnética funcional (fMRI) ha desempeñado un papel fundamental en la comprensión de la complejidad de este órgano asombroso. A lo largo de las últimas décadas, la fMRI se ha convertido en una herramienta invaluable que ha permitido explorar las intricadas redes neuronales y comprender cómo se traducen los procesos mentales en actividad cerebral observable.
	
	El viaje de la fMRI en la investigación cerebral ha sido testigo de un progreso excepcional. Desde sus primeras aplicaciones que proporcionaban una visión más detallada de las regiones activas durante tareas específicas, hasta su uso actual en la exploración dinámica de la conectividad funcional y la plasticidad cerebral, la fMRI ha revolucionado nuestra capacidad para desentrañar los misterios del cerebro humano.
	
	La literatura científica está repleta de estudios que han utilizado la fMRI para investigar una amplia gama de fenómenos cerebrales. Desde la percepción sensorial hasta la toma de decisiones, pasando por la memoria y las funciones ejecutivas, la fMRI ha proporcionado una ventana única para observar la actividad cerebral en tiempo real. La capacidad de visualizar la actividad cerebral mientras los participantes realizan tareas cognitivas específicas ha permitido a los investigadores hacer avances significativos en la comprensión de la base neural de la cognición y el comportamiento.
	
	Este trabajo explora el rico panorama de estudios que han empleado la fMRI para adentrarse en los rincones más profundos de la mente humana. Desde investigaciones que han desafiado nuestras nociones sobre la especialización funcional de regiones cerebrales hasta aquellas que han examinado la lateralización hemisférica y la plasticidad neuronal, la fMRI ha sido una herramienta versátil que ha revelado una amplia gama de conocimientos sobre el cerebro humano. En este contexto, este estudio se suma a la creciente narrativa que utiliza la fMRI como un medio para profundizar nuestra comprensión de la función cerebral y su relación con la cognición.
	
	\subsection{4}
	
	Introducción:
	
	La lateralización hemisférica, un fenómeno fascinante del cerebro humano, ha capturado la atención de los investigadores y científicos cognitivos durante décadas. Este concepto se refiere a la asimetría funcional y estructural entre los hemisferios cerebrales derecho e izquierdo, destacando la notable especialización de cada uno en ciertas funciones cognitivas y procesos mentales.
	
	Desde los primeros estudios sobre lateralización hemisférica, se ha observado que el hemisferio izquierdo tiende a estar más involucrado en funciones relacionadas con el lenguaje, el procesamiento verbal y la lógica, mientras que el hemisferio derecho se especializa en tareas visoespaciales, procesamiento emocional y creatividad. Sin embargo, esta dicotomía no implica que una mitad del cerebro sea exclusivamente responsable de ciertas actividades; más bien, ambas mitades colaboran de manera compleja para dar lugar a la experiencia humana integral.
	
	El entendimiento de la lateralización hemisférica se ha enriquecido a lo largo del tiempo gracias a diversas metodologías de investigación, desde estudios de lesiones cerebrales hasta técnicas de neuroimagen avanzadas. Las investigaciones han revelado patrones sutiles y complejas interacciones entre los hemisferios, desafiando en ocasiones las ideas preconcebidas y sugiriendo una plasticidad cerebral que permite adaptarse a las demandas cambiantes del entorno.
	
	Este estudio se adentra en el fascinante terreno de la lateralización hemisférica, utilizando la resonancia magnética funcional (fMRI) como herramienta principal. La fMRI proporciona una ventana única para observar la actividad cerebral en tiempo real, permitiendo investigar cómo diferentes regiones cerebrales y hemisferios contribuyen a diversas funciones cognitivas. Al explorar la lateralización hemisférica con esta técnica, pretendemos aportar nuevos matices a nuestro entendimiento de la organización cerebral y su influencia en la cognición humana. Este trabajo se suma al continuo esfuerzo por desentrañar los misterios de la lateralización hemisférica, brindando una contribución significativa al campo de la neurociencia cognitiva.
	
	-------
	La percepción visual, esencial para nuestra comprensión del entorno, se ha convertido en un área central de estudio en la neurociencia cognitiva. Entre las diversas facetas de la percepción visual, la lateralización hemisférica del cerebro ha emergido como un fenómeno intrigante, especialmente en relación con la percepción global y local de estímulos visuales.
	
	La percepción visual se inicia en la vía visual primaria, donde la información visual captada por los ojos es transmitida al cerebro a través del nervio óptico. Esta vía visual primaria, también conocida como la corteza visual primaria (V1), desempeña un papel fundamental en el procesamiento inicial de la información visual. Aquí, las características fundamentales de la estimulación visual, como la orientación y la frecuencia espacial, son detectadas y codificadas.
	
	La lateralización hemisférica en la percepción visual se manifiesta en la asimetría funcional entre los dos hemisferios cerebrales, donde el hemisferio derecho tiende a especializarse en la percepción global, capturando la esencia general de un estímulo visual, mientras que el hemisferio izquierdo se inclina hacia la percepción local, centrándose en detalles específicos.
	
	Un elemento clave en esta dicotomía perceptual es la frecuencia espacial, que se refiere a la densidad de cambios en la luminancia de una imagen. Las frecuencias espaciales más bajas están asociadas con patrones más amplios y globales, mientras que las frecuencias más altas representan detalles más finos y locales. Esta relación entre la lateralización hemisférica y la percepción global/local se basa en cómo cada hemisferio procesa y prioriza las distintas frecuencias espaciales, contribuyendo a nuestra comprensión de la escena visual.
	
	En este contexto, la presente tesis se sumerge en el estudio de la lateralización hemisférica en la percepción visual, explorando cómo los hemisferios cerebrales procesan y interpretan información visual, particularmente en relación con la frecuencia espacial y la dicotomía global/local. Al utilizar la resonancia magnética funcional (fMRI), esta investigación busca desentrañar los mecanismos que subyacen a la percepción visual y ampliar nuestro entendimiento de cómo el cerebro organiza y da sentido a la riqueza visual que nos rodea.
	
	\subsection{5}
	
	Introducción:
	
	En el fascinante viaje de explorar la lateralización hemisférica y su vínculo con la percepción visual, se hace imperativo contar con un marco conceptual sólido que guíe nuestra investigación. Aquí es donde los tres niveles propuestos por David Marr —computacional, algorítmico e implementación— se revelan como herramientas fundamentales para desentrañar los misterios que yacen en la organización neuronal de la percepción visual.
	
	En el nivel computacional, nos sumergimos en la comprensión de los objetivos y la lógica subyacente en la lateralización hemisférica y la percepción global/local. ¿Por qué el cerebro muestra preferencia por ciertos aspectos visuales? ¿Cómo se logra la integración de información global y local? Estas preguntas computacionales proporcionan el punto de partida para entender la función de los hemisferios cerebrales en el procesamiento visual.
	
	El nivel algorítmico se erige como el siguiente escalón, donde nos abocamos a desglosar los procesos que podrían llevar a la realización de los objetivos computacionales. Aquí, nos adentramos en los mecanismos precisos que los hemisferios cerebrales emplean para seleccionar y procesar información visual, explorando las estrategias algorítmicas que subyacen a la percepción global/local y su relación con la frecuencia espacial.
	
	Finalmente, en el nivel de implementación, la atención se centra en la materialización biológica y neuronal de estos algoritmos. ¿Cómo se traducen los procesos algorítmicos en cambios y patrones observables en la actividad cerebral? La resonancia magnética funcional (fMRI) emerge como una herramienta valiosa, permitiéndonos mapear la actividad cerebral y vincularla directamente con los fenómenos observados en los niveles computacional y algorítmico.
	
	Al emplear la estructura propuesta por Marr, esta investigación no solo busca comprender la lateralización hemisférica y la percepción visual desde una perspectiva integral, sino que también aspira a proporcionar un marco sólido para la interpretación de los resultados obtenidos. Al combinar la riqueza conceptual de los niveles de Marr con las capacidades de la fMRI, buscamos desentrañar los misterios de cómo el cerebro humano organiza y procesa la información visual, ofreciendo una contribución significativa al campo de la neurociencia cognitiva.
	
	\subsection{6}
	
	Introducción:
	
	En la búsqueda constante por comprender los intrincados procesos que rigen la percepción visual y la lateralización hemisférica en el cerebro humano, las herramientas de investigación desempeñan un papel crucial. En este contexto, los Campos Receptivos Poblacionales (pRF) emergen como una herramienta esencial para desvelar los secretos de la activación cerebral en cada hemisferio.
	
	Los pRF, o campos receptivos poblacionales, representan una ventana única a la actividad neuronal, permitiendo la cuantificación de la respuesta neuronal a estímulos visuales específicos. Estos campos receptivos no solo capturan la respuesta de neuronas individuales, sino que, de manera ingeniosa, consideran la población de neuronas que responde a un estímulo particular. Así, los pRF se convierten en una herramienta poderosa para mapear la organización espacial de la activación cerebral en respuesta a estímulos visuales.
	
	En términos más concretos, un pRF es la región del espacio visual que activa una unidad neuronal específica. Cuando se expande esta noción a la escala poblacional, obtenemos una representación detallada de cómo diferentes áreas cerebrales, y específicamente cada hemisferio, responden a estímulos visuales variados. La activación resultante se convierte en un marcador invaluable para entender cómo el cerebro procesa, organiza y extrae información visual.
	
	En esta investigación, aprovechamos la capacidad única de los pRF para proporcionar un mapeo detallado de la activación hemisférica en respuesta a estímulos visuales específicos. Al emplear esta herramienta, no solo aspiramos a visualizar la lateralización hemisférica en acción, sino también a identificar patrones distintivos que caracterizan la percepción visual en cada hemisferio. Los pRF se erigen así como una llave maestra para desbloquear las complejidades de la organización neural en el estudio de la percepción visual y la lateralización hemisférica.
	
	-----
	
	Introducción:
	
	En el fascinante terreno de la investigación cerebral, la comprensión de la percepción visual y la lateralización hemisférica requiere herramientas avanzadas que desentrañen los secretos de la actividad cerebral. Dentro de este contexto, los Campos Receptivos Poblacionales (pRF) surgen como una herramienta fundamental, especialmente cuando se integran con la resonancia magnética funcional (fMRI).
	
	Los pRF, o campos receptivos poblacionales, actúan como ventanas precisas hacia la actividad neuronal, permitiendo cuantificar la respuesta a estímulos visuales específicos. Lo notable de los pRF es su capacidad para no solo capturar la respuesta de neuronas individuales, sino también para considerar la población de neuronas que reacciona a un estímulo en particular. Esta característica los convierte en herramientas potentes para mapear la organización espacial de la activación cerebral en respuesta a estímulos visuales.
	
	Cuando se integran con la fMRI, los pRF potencian aún más su utilidad. La fMRI nos brinda una visión panorámica de la actividad cerebral al detectar cambios en el flujo sanguíneo relacionados con la actividad neuronal. Combinada con los pRF, esta técnica nos permite no solo observar la respuesta neuronal, sino también cartografiar cómo diferentes áreas cerebrales, incluidos ambos hemisferios, se activan frente a estímulos visuales específicos.
	
	En nuestra investigación, aprovechamos la potencia combinada de los pRF y la fMRI para obtener un mapeo detallado de la activación hemisférica en respuesta a estímulos visuales. Esta integración nos permite visualizar no solo la lateralización hemisférica en acción, sino también identificar patrones distintivos que caracterizan la percepción visual en cada hemisferio. Así, los pRF y la fMRI se erigen como aliados clave para desentrañar las complejidades de la organización neural y ofrecer una visión más profunda de la percepción visual y la lateralización hemisférica.
	
	-----
	
	Introducción:
	
	En el intrigante ámbito de la investigación cerebral, la comprensión de la percepción visual y la lateralización hemisférica requiere herramientas avanzadas para desentrañar los secretos de la actividad cerebral. Entre estas herramientas, los Campos Receptivos Poblacionales (pRF) emergen como modelos matemáticos fundamentales, especialmente cuando se integran con la resonancia magnética funcional (fMRI).
	
	Los pRF, o campos receptivos poblacionales, son modelos matemáticos que actúan como ventanas precisas hacia la actividad neuronal, permitiéndonos cuantificar la respuesta a estímulos visuales específicos. La notable característica de los pRF es su capacidad para no solo capturar la respuesta de neuronas individuales, sino también para considerar la población de neuronas que reacciona a un estímulo en particular. Este enfoque modelístico se convierte en una herramienta poderosa para mapear la organización espacial de la activación cerebral en respuesta a estímulos visuales.
	
	Cuando se integran con la fMRI, los pRF potencian aún más su utilidad. La fMRI nos brinda una visión panorámica de la actividad cerebral al detectar cambios en el flujo sanguíneo relacionados con la actividad neuronal. Combinada con los modelos matemáticos de los pRF, esta técnica nos permite no solo observar la respuesta neuronal, sino también cartografiar cómo diferentes áreas cerebrales, incluidos ambos hemisferios, se activan frente a estímulos visuales específicos.
	
	En nuestra investigación, aprovechamos la potencia combinada de los pRF y la fMRI para obtener un mapeo detallado de la activación hemisférica en respuesta a estímulos visuales. Esta integración nos permite visualizar no solo la lateralización hemisférica en acción, sino también identificar patrones distintivos que caracterizan la percepción visual en cada hemisferio. Así, los pRF, como modelos matemáticos, y la fMRI se erigen como aliados clave para desentrañar las complejidades de la organización neural y ofrecer una visión más profunda de la percepción visual y la lateralización hemisférica.
	
	\subsection{7}
	
	Introducción:
	
	El estudio del cerebro, con un enfoque particular en la percepción visual y la lateralización hemisférica, se erige como un campo de investigación crucial en el entendimiento de la complejidad de la mente humana. La importancia de este estudio radica en su capacidad para arrojar luz sobre los intrincados procesos cognitivos que gobiernan la manera en que percibimos el mundo que nos rodea.
	
	La percepción visual, siendo una piedra angular de la experiencia humana, desempeña un papel vital en la formación de nuestra comprensión del entorno. Comprender cómo el cerebro procesa y organiza la información visual no solo contribuye a la base científica del conocimiento, sino que también tiene implicaciones significativas en campos como la psicología, la neurociencia y la inteligencia artificial.
	
	La lateralización hemisférica agrega una capa adicional de complejidad a este estudio. La observación de que los hemisferios cerebrales desempeñan roles distintos en la percepción visual abre la puerta a una comprensión más profunda de cómo la estructura cerebral influye en la interpretación y procesamiento de la información visual. Esta asimetría funcional entre el hemisferio derecho y el izquierdo destaca la necesidad de explorar cómo estos desempeñan papeles complementarios o contrastantes en la formación de la percepción visual.
	
	En este contexto, el presente estudio se propone investigar la lateralización hemisférica en relación con la percepción visual, empleando campos receptivos poblacionales (pRF) como herramientas clave. La importancia de esta investigación radica en su potencial para proporcionar conocimientos esenciales sobre la organización neural detrás de la percepción visual y cómo esta puede variar entre los hemisferios cerebrales. Estos hallazgos no solo enriquecerán nuestra comprensión de la mente humana, sino que también pueden tener aplicaciones prácticas en el diseño de intervenciones terapéuticas y tecnologías de vanguardia. En última instancia, este estudio contribuirá al continuo avance de nuestro conocimiento sobre el cerebro y su papel fundamental en la percepción y la cognición.
	
	\subsection{8}
	
	Introducción:
	
	El estudio del cerebro y sus funciones complejas ha experimentado un extraordinario avance gracias a la convergencia de la neurociencia y la informática. En particular, la utilización de diversas herramientas computacionales ha ampliado nuestras capacidades para analizar y comprender los datos provenientes de la resonancia magnética funcional (fMRI), una técnica fundamental en la investigación neurocientífica.
	
	El presente estudio se sitúa en la intersección entre la neurociencia y la informática, buscando desentrañar los misterios de la lateralización hemisférica y la percepción visual. Para llevar a cabo esta tarea, se recurre a un conjunto variado de herramientas computacionales que desempeñan roles cruciales en diferentes etapas del proceso de investigación.
	
	En el preprocesamiento de los datos de fMRI, las herramientas computacionales permiten realizar correcciones y transformaciones necesarias para garantizar la calidad y confiabilidad de los resultados. Estos procedimientos, que van desde la eliminación de artefactos hasta la normalización espacial, son fundamentales para obtener datos limpios y coherentes.
	
	La estimación de parámetros de campos receptivos poblacionales (pRF) constituye otro aspecto clave de este estudio. Aquí, herramientas computacionales especializadas facilitan la modelación matemática de los campos receptivos, permitiendo una caracterización precisa de la actividad neural en respuesta a estímulos visuales. Esta fase no solo mejora la interpretación de los resultados sino que también contribuye a la generación de modelos más realistas.
	
	La definición de áreas cerebrales de interés y la aplicación de pruebas estadísticas para la evaluación de hipótesis representan los pilares finales de este estudio. Mediante herramientas computacionales avanzadas, se delimitan áreas específicas del cerebro relacionadas con la lateralización hemisférica y la percepción visual. Además, se aplican rigurosos análisis estadísticos para validar y fundamentar las conclusiones extraídas de los datos recopilados.
	
	En conjunto, la integración de estas herramientas computacionales no solo potencia la capacidad de explorar fenómenos cerebrales complejos, sino que también allana el camino para descubrimientos significativos en el entendimiento de la mente humana. Este enfoque interdisciplinario refleja la sinergia entre la neurociencia y la informática, promoviendo avances significativos en la comprensión de la función cerebral y sus implicaciones en la percepción visual y la lateralización hemisférica.
	
	\subsection{DFF}
	
	A pesar de la aparente similitud de los dos hemisferios cerebrales en el cerebro humano, existen asimetrías significativas en sus funciones. En la mayoría de las personas, el hemisferio izquierdo (HI) desempeña un papel primario en la producción y percepción del lenguaje. Esta función se evidencia claramente en déficits del lenguaje y del habla que suelen acompañar al daño en este lado del cerebro. De manera similar, los déficits perceptuales y de atención son generalmente más notables después del daño al hemisferio derecho (HD). Las teorías iniciales tendían a centrarse en dominios de tarea, enfatizando que el HI y el HD estaban especializados en el procesamiento de información lingüística y espacial, respectivamente.
	
	Sin embargo, con un estudio más cuidadoso, quedó claro que tales dicotomías simples no capturaban la complejidad de la función cerebral. Un estudio clave involucró a participantes que observaron letras jerárquicamente estructuradas, presentadas en el campo visual izquierdo (CVI) o derecho (CVD). Los resultados mostraron una ventaja del CVD (HI) en el tiempo de respuesta cuando el objetivo a identificar estaba definido por la forma local del componente, y una ventaja del CVI (HD) cuando estaba definido por la forma global. La hipótesis sostenía que esta asimetría indicaba que los hemisferios diferían en la sensibilidad a la información visual en el estímulo, con el HD respondiendo más eficientemente a información de baja frecuencia espacial y el HI a información de alta frecuencia espacial.
	
	Esta asimetría se basa en la idea de que el hemisferio contralateral al estímulo domina el procesamiento. Investigaciones en pacientes con daño unilateral al HD mostraron déficits no solo en dibujar objetos globales, sino también en la percepción y memoria de información global. A la inversa, los pacientes con daño unilateral al HI mostraron un rendimiento deficiente en objetos locales. La teoría del Doble Filtro Frecuencial (DFF) propone dos etapas de filtrado sucesivas basadas en mecanismos de procesamiento visual sintonizados a la frecuencia espacial. Ambos hemisferios tienen acceso a la información relevante para la tarea, pero el filtrado asimétrico resulta en representaciones no idénticas.
	
	La teoría DFF proporciona un modelo integrado y computacional para explicar una amplia variedad de hallazgos en la literatura sobre lateralidad. No se centra en dominios de tarea específicos ni implica una dicotomía fuerte entre las funciones del HI y el HD. Esta teoría podría ser útil para entender otros dominios de tarea donde los efectos de lateralidad son notables. En resumen, las asimetrías hemisféricas en la percepción visual se deben a la eficiencia diferencial en la representación de información relevante para distintas tareas.
	
	La teoría del Doble Filtro Frecuencial (DFF) propone un marco conceptual para entender las asimetrías hemisféricas en la percepción visual, específicamente en relación con la sensibilidad a la información de frecuencia espacial en estímulos visuales complejos.
	
	La teoría postula dos etapas sucesivas de filtrado en el procesamiento visual, basadas en mecanismos de afinidad por la frecuencia espacial. En la primera etapa, se realiza una selección de la gama de frecuencias espaciales relevantes para la tarea en cuestión. Se asume que esta etapa es realizada de manera simétrica por ambos hemisferios.
	
	La salida de esta primera etapa sirve como entrada para la segunda etapa de procesamiento. En esta fase, aparecen las asimetrías hemisféricas. Cada hemisferio actúa como un filtro en la información seleccionada de la primera etapa. El hemisferio derecho (HD) opera como un filtro de paso bajo (low-pass), permitiendo que más información de baja frecuencia espacial pase para su procesamiento adicional. En contraste, el hemisferio izquierdo (HI) actúa como un filtro de paso alto (high-pass), facilitando el paso de más información de alta frecuencia espacial.
	
	La teoría del DFF sostiene que ambos hemisferios tienen acceso a la información relevante para la tarea, pero la asimetría en el filtrado de la información inicialmente seleccionada resulta en representaciones no idénticas. Esto implica que cada hemisferio es más eficiente para ciertos tipos de tareas, dependiendo de si se requiere un énfasis en el procesamiento de detalles locales (alta frecuencia espacial) o en la captura de patrones globales (baja frecuencia espacial).
	
	En resumen, la teoría DFF ofrece una explicación para las diferencias en la percepción visual entre los hemisferios cerebrales, destacando la importancia de la frecuencia espacial en la organización de la información visual. Esta teoría no solo ayuda a comprender la lateralización hemisférica en la percepción visual, sino que también proporciona un marco conceptual para explorar la eficiencia relativa de cada hemisferio en diversas tareas visuales.
	
	\section{Codigo}
	
	Las claves (`KeysViewHDF5`) que has proporcionado son atributos o conjuntos de datos que pueden estar presentes en un archivo de resultados de fMRI guardado en formato HDF5 (usado comúnmente para archivos de MATLAB v7.3 y posteriores). A continuación, se explica cada uno de ellos:
	
	1. **R2**: Este valor generalmente representa el coeficiente de determinación (R cuadrado), que mide la proporción de la variabilidad en los datos que puede explicarse por el modelo.
	
	2. **R2run**: Puede ser la R cuadrado específica de una ejecución o escaneo particular.
	
	3. **SNR**: Relación señal-ruido.
	
	4. **SNRafter**: SNR después de algún proceso o manipulación específica.
	
	5. **SNRbefore**: SNR antes de algún proceso o manipulación específica.
	
	6. **hrffitvoxels**: Datos relacionados con la estimación de la función de respuesta hemodinámica (HRF) para ciertos voxels.
	
	7. **inputs**: Datos de entrada utilizados en el análisis.
	
	8. **meanvol**: Volumen medio.
	
	9. **modelmd**: Resultados del análisis de modelo.
	
	10. **models**: Modelos utilizados en el análisis.
	
	11. **modelse**: Errores estándar asociados con el modelo.
	
	12. **noise**: Datos relacionados con el ruido.
	
	13. **noisepool**: Piscina de ruido.
	
	14. **pcR2**: Puede referirse a la R cuadrado después del preprocesamiento.
	
	15. **pcR2final**: R cuadrado final después del preprocesamiento.
	
	16. **pcnum**: Número de componentes principales.
	
	17. **pcregressors**: Regresores obtenidos a partir del análisis de componentes principales.
	
	18. **pcvoxels**: Voxeles relacionados con el análisis de componentes principales.
	
	19. **pcweights**: Pesos asociados con el análisis de componentes principales.
	
	20. **signal**: Datos relacionados con la señal.
	
	Es importante tener en cuenta que la interpretación precisa de estos datos puede depender del contexto específico del análisis fMRI que se haya realizado y del software utilizado para generar estos resultados. Además, algunos de estos atributos pueden estar relacionados con análisis específicos o procesos de preprocesamiento realizados en los datos originales.
	


\end{document}